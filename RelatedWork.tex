\chapter{Related Work}
\label{chapter:related-work}

Existing sketch tools can be classified into two categories: (1) those that interpret sketches with the purpose of turning them into formal objects, and (2) those that support sketching activities in general. The survey on sketch-based systems by Johnson et. al. \citep{Johnson} provides a more extensive look of all tools created to date than what we necessarily can provide in the blow.

\section{Tools that Interpret}
\label{relatedwork:1}

The  tools within this category interpret a digital freeform sketch and then generate a formal representation. While we purposefully chose not to interpret sketches in Calico, these tools remain relevant as background work.

Many tools that interpret sketches focus on generating prototypes for user interfaces. One of the earliest  such tools is SILK \citep{Landay}, which allowed users to generate a usable GUI from a sketch, and then generate the corresponding Java code to implement that GUI. SILK was the first such tool of its kind, which stressed both the importance of working in a low-fidelity environment, but enabled rapid iterations of mockups by generating usable simulations to give near immediate feedback. SILK differed greatly from Calico in that it helped designers iterate by providing feedback, while Calico encourages developers by making sketches easier to manipulate using scraps. Another tool, DENIM \citep{newman2003denim}, allows users to create low fidelity mockups of websites and then run simulations of that website. DENIM built on the concepts of SILK by both supporting interpretation of sketched content and navigation between webpages. DENIM contributed the notation of multiple levels of hierarchy for navigating sketches, in which the user zoomed out to navigate the network of websites, and zoomed in to view sketched content in webpages. DENIM's use of multiple levels of levels of abstraction for navigating webpages, and linking webpages, is similar to Calico's intentional interfaces, but is highly specialized for designing webpages. Other tools targeted multiple domains, such as the tools InkKit and Freeform, which created usable software systems from sketches \citep{chung2005inkkit,Plimmer}. These family of tools stressed the importance of retaining the sketchy look-and-feel of sketches, which was based on their insight that visually beautifying content too early can be harmful to the designer \citep{Shipman}. Unlike previous tools, InkKit and Freeform interpreted different types of sketches, such as user interface and ER diagrams, which they used to generate a working prototype system. By interpreting sketches from multiple perspectives, these systems provide rich prototypes for feedback at a low cost of effort for the designer. These systems are similar to Calico in that users can sketch a system from multiple perspectives, however these tools restrict users to using a pre-determined set of notations.

Another group of interpretive sketch systems includes those that recognize UML elements. The earliest such tool was Knight \citep{damm2000tool}, which allowed  designers to create partial UML diagrams by selecting a portion or all of their sketch to be interpreted as UML elements. Knight was inspired by studying software designers at the whiteboard, and contributed the observation that software sketches begin vague, but become refined over time. The Knight tool allowed designers to sketch, and covert their sketches into a beautified UML representation by manually triggering it. Both Calico and Knight allow users to refine their drawings from plain whiteboard sketches into objects at a later point, however Knight only interprets UML class and sequence diagram notations. The tool SUMLOW \citep{chen2008sumlow} further advanced on the work of Knight by allowing interpreted sketches to retain their sketchy appearance after interpretation (citing InkKit as inspiration), and allowed the mixing of different UML notations. While both Knight and SUMLOW made UML diagrams a less viscous representations by allowing users to simply sketch additions, the retainment of the sketchy appearance after interpetation introduced another step of provisionality to sketches, which was not available to software designers in other sketch interpretation tool. Both Calico and SUMLOW are similar in this philosophy. Hosking and Grundy later integrated this tool into a full software design environment, Marama-Sketch \citep{Grundy}, which designers can use to sketch diagrams in a broad variety of notations. Marama sketch brought tighter integration between sketched models and actual code, a quality which Calico itself does not have. Numerous other UML-oriented sketch tools have been created, generally following similar strategies (for a survey, see \citep{Johnson}). 

Another approach taken was building a general purpose framework for interpreting sketches. LADDER was such a tool that provided an extensible, general purpose language which could be adapted to many notations, including software design notations, biology and chemistry notations, and military symbols using maps \cite{hammond2006ladder}. Of the many approaches, most target a single, or small set of notations. Dixon's work provided an approach that attempted to unify different interpretation engines by providing a framework that queries multiple engines simultaneously, and ranks the results by their probability of success \cite{dixon2008whiteboards}. These tools vary greatly from Calico given that they are domain specific, while Calico remains a domain-generic environment.

Other tools allow mixed elements to coexist in the same area as well, though interpretation is left to explicit manual choice by the user rather than automated recognition. An example of such a system is Patches, in which the user can create sets of irregular shapes called ``patches'' \citep{Kramer}. These can be moved around, be made translucent, and, most importantly, the patch can be assigned an interpretation where it can become a list, outline, table, or other element. The functionality, as well as look-and-feel, of the scraps functionality in Calico is fundementally very similar to the Patches system. However, they differ in their goal, where Patches viewed its functionality as providing lenses over existing sketches, and provided transparencies for annotations over existing sketches such as architecture diagrams. Calico, on the other hand, uses scraps as building blocks to build representations within the domain of software. Kramer's work later inspired the overall functionality of the sketch framework SATIN \citep{Hong}, which was later used to create the tools DENIM (previously mentioned) and Designer's Outpost (mentioned in the next section). 

\section{Systems that Support Sketching Activities}
\label{relatedwork:2}

The second type of sketch system is aimed at supporting the general design activity. These systems help organize sketched artifacts during meetings, make them retrievable, and help manage and maintain the plethora of sketched design content that gets generated during design sessions. 

The very first of these systems were created for the Liveboard system at PARC. Colab was the first system to work with this hardware, and acted as a meeting tool \citep{Stefik}. It had a simple, uncluttered interface with a mode selector and a visible set of saved sessions in the fashion of a filmstrip. The next system developed for the Liveboard was Tivoli \citep{Pederson}, which used the filmstrip metaphor to manage sketches in a side panel. Tivoli also organized sketches into chunked items for easy moving, and had additional functionality for creating empty space. Tivoli did not attempt to recognize any of the drawn elements, but rather provided intelligent support for common tasks, such as creating lists and checking off items in lists. The clustering of items and use of lists in Tivoli is similar to the functionality that scraps and list scraps provide. After Tivoli, Dolphin \citep{Streitz:1994:DIM:192844.193044} introduced a client-server architecture for group collaboration and allowed users to create links between canvases. Dolphin was the first whitebaord system to provide hyperlinks between content in canvases, which would not appear again in other tools until the creation of DENIM ten years later. Hyperlinks in Dolphin provide more flexibility than intentional interfces for organizing lists, but Dolphin did not provide a visualization for the organizing of canvases, such as Calico's cluster view. A significant difference between these systems and Calico lies in intentional interfaces, which provides a spatial orientation of sketches as compared to the paginated style, which leads to older sketches moving around. 

Many other tools provide methods for managing the presence of multiple sketches within a fixed space. PostBrainStorm \citep{guimbretiere2001fluid} uses the border areas as areas where sketches auto-shrink, thereby minimizing space use by elements that are temporarily set aside. PostBrainstorm presented an exploration of an ubiqutious computing approach to whiteboard software and hardware, allowing users to place a physical picture on a whiteboard, and have the same picture automatically appear in the system. While Calico did not experiment in the domain of ubiquitous computing, the importing of existing photos and documents were important features in the use of Calico as well. Flatland \citep{mynatt1999flatland} implemented a variant of this last idea by automatically grouping sketches into clusters, and shrinking and moving them out of the way when more space is needed. Flatland was not the first to provide clusters, but Flatland contributed a rich exploration of it's capability for specialized support, including organizing lists, creating maps, etc., and allowing users to move backwards and forwards in the history of clusters. Scraps within Calico, in comparison, do not support as many domains, but instead are stackable and relate-able. Range \citep{Ju} extended the approach taken by Flatland by using a person's physical proximity to the board as a trigger to create space, and also presented past sketches back to the designer when they were physically far away from the board in order to provide inspiration for creative work. Overall, many of these systems attempt to support the designer at the whiteboard in general activities such as clustering objects and making space without the designer manually requesting this space. Calico does not provide such ubiquitous support in a single whiteboard, but instead Calico allows users to manually organize their sketches using intentional interfaces because manually organizing spaces has value in one's own design process.

A third set of tools in this second type specifically target the organization of multiple sketches and canvases. Bellamy et al. \cite{Bellamy:2011:STI:1985793.1985909} presented an ideation tool for building user interfaces. Users could sketch the states of a user interface, and the sketch presented a set of storyboards back to the user. Additionally, the history of the user's exploration was presented back to the user as a tree that traced the undo history. Other systems targeted the visualization of separate canvases, such as IdeaVis \cite{geyer2012ideavis}, which presented a visualization for a network of whiteboard canvases. The IdeaVis system captures hand-drawn sketches using a camera, and presents them back to a co-located sketch team using an ``interactive hyperbolic tree visualization'', in which users can navigate a network of canvases, and the visual size of canvases become larger when in close proximity to the active canvas. Their studies showed that automatically making relevant canvases large helped users maintain focus to important sketches. This system is most similar to Calico's intentional interfaces in that they both visualize a network of canvases, however intentional interfaces maintains a fixed size for canvases.

Another set of tools targeted the collaboration aspect of electronic whiteboard systems. Designer's Outpost \citep{klemmer2001designers} bridged the gap between physical Post-It notes and digital content, transitioning content of Post-It notes into digital artifacts that then could be organized in various ways. The core functionality of this system allowed users to create links between digital Post-It notes, group Post-It notes into clusters, and zoom in and expand individual Post-It notes to write in great detail. The designer's outpost system provided novel methods of interacting with remote users, such as as showing a user's shadow on the whiteboard in order to communicate gestures and body language, and using a pointer. Calico, in comparison, does not communicate body language, but instead uses the fading highlighter to communicate transient information. TEAM STORM \citep{Hailpern} presented another approach, which addressed the issue of public and private spaces. Users could create sketches on their own devices, and had the option of sharing their sketches in the public space, in which other users could interact with and provide feedback on sketches. Further building on the notation of a shared collaborative space, the GAMBIT \cite{Sangiorgi:2012:GAM:2305484.2305527} system addressed the issue of using multiple devices together, including large displaces, laptops, tabletops, and phones. In studies of GAMBIT, researchers observed a trend in sessions in which users first build mental models of their domain, construct scenarios, and lastly sketch prototypes using GAMBIT. The notation of public and private spaces using GAMBIT was viewed as helpful to users. In comparison, sketches in Calico are always public, but users can move to a separate canvas to work asynchronously. Lastly, Camel \cite{cataldo2009camel} integrated a whiteboard system into a formal Eclipse UML environment, allowing users to post UML diagrams onto poster boards. By integrating into a formal environment, Camel provided software teams value by supporting code reviews and design walkthroughs which were very closely tied to a team's software code. 

%%% Local Variables: ***
%%% mode: latex ***
%%% TeX-master: "thesis.tex" ***
%%% End: ***
