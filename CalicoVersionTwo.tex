\chapter{Calico Version Two}
\label{chapter:calico-version-two}

Over the past six years, we have gone through an iterative process of designing, implementing, and evaluating various incarnations of Calico. Through various experiments and trials with the technology over several years, including using the tool ourselves, deploying it in the classroom, and formal experimentation \cite{mangano2012design}, we have slowly, but surely developed an understanding of how we can mesh more advanced functionality with the fluidity that the whiteboard experience demands. We have specifically refined what first was a broad range of exploratory features into a small, cohesive set of just five features that we propose here in this topic proposal. We believe these five features, some of which we have already built, some of which we are currently designing, to be effective in supporting the fourteen behaviors that we identified in Section \ref{chapter:motivation}. In the below we briefly introduce the features and how they relate to each other, with subsequent discussions in Section \ref{chapter:calico-version-two:implementation}, providing a more detail discussion of how the features work and are implemented. What is important here is how the set of features collectively address the set of behaviors, which is summarized in Table \ref{table:calico-version-two:designbehaviors}. 

\section{Features}
\label{chapter:calico-version-two:features}

The basis for all of the features in Calico is scraps and connectors. Scraps are the mechanism by which users interact with the sketches they create and, along with connectors, provide the building blocks for users to create representations for different kinds of diagrams. Taking inspiration from the Patches \cite{Kramer}, Calico uses the common user action of circumscribing drawn content with the pen \cite{Hendry} to create a scrap, and promotes the circumscribed content as belonging to the newly created scrap (see Figure 1a). Scraps, in being manipulatable objects, can be moved, copied, resized, deleted, etc. Furthermore, scraps exhibit three key properties that make them so essential to our work. First, they retain their shape from the pen-stroke that created them. This gives the user flexibility to represent the different kinds of diagrams that designers create, since shape is a fundamental quality of notations \cite{moody2009physics}. Second, they are automatically grouped by stacking them on top of one another. Much of what software designers draw is hierarchical, which is often visually represented by containment \cite{Moody:2009:TDV:1685992.1686136}. Scrap interactions, and similarly all other features that build on scraps, implicitly recognize the grouping by behaving as though a scrap and all other scraps stacked on top of it are a single object; operations such as move, copy, etc. operate on them as though they were a single unit. Lastly, users can create connectors that relate scraps to one another. Connectors provide a fundamental complement to scraps, allowing users to create diagrams that are not only visually similar to the box-and-arrow diagrams that are present in a great many of software design notations \cite{Shaw:1993:PCA:645540.657852}, but also behave like box-and-arrow diagrams too: the connectors visually follow a scrap as it moves. The connector itself further provides a fundamental component that other features can build on by adding different arrowheads, cardinality, and labels. 

Intentional interfaces are a mechanism we propose to relate multiple sketches together. A designer working on a design problem will need to organize their sketches across multiple canvases. The intentional interfaces feature helps the designer in this activity by using lightweight rational capture for organizing and navigating their sketches. The feature does this in three ways. First, it maintains a history of the designer’s exploration of the design space by generating canvas links. Second, canvases are tagged with their purpose in the design exploration. Upon entering a canvas for the first time, the designer is asked to select a tag that summarizes their intention for creating that canvas, such as ``alternative'', ``perspective'', or ``level of abstraction''. This tag later serves the designer by allowing them to explicitly move between alternatives or abstractions. Third, Calico uses the links and tags to build the intention view, visible in Figure 1c, which displays the emerging network of sketches. In this perspective, the designer can review their design and design progress as a whole, select any canvas to work in, arrange canvases, and change their links. 

The palette complements the previous features by making scraps reusable. The palette serves as a global source to save scraps, which can immediately be reused anywhere by dragging it back to the canvas from the palette. While not an innovative feature in itself, since palettes are an integral part of any drawing program, the palette serves an important role in the practicality of the previous features. First, the palette is user driven. The palette initializes as an empty set that the designers may fill with their own vocabulary of icons. The palette allows the user to toggle between sets if they wish to create multiple vocabularies. Second, assembling a custom vocabulary would not be helpful if it were limited to a single canvas, and thus the palette allows its contents to be reused across canvases. The global property of the palette allows this vocabulary to be reused everywhere. Third, the palette not only stores the visual structure of scraps, but also the behaviors assigned to them. This is a key supporting feature for compositional notations, as recreating a particular kind of scrap with certain visual expressiveness and behavior should not be necessary. The palette allows the designer to create a UML, ER, or impromptu notation, and reuse it indefinitely.

Lastly, the highlighter separates the act of highlighting from drawing. This functionality allows the designer to point, circle, and draw sequences over diagrams, but the strokes made by the designer fade a few seconds after they are made. While a small piece of functionality, the highlighter serves an important role. It allows a group of designers, and even the individual designer, to perform mental simulations over a sketch to explain them to each other, both while at the same whiteboard and over the network in distributed cases.

\begin{figure}%
  \centering
  \subfigure[Scraps] {
      \label{fig:calico-version-two:overviewa}
	  \includegraphics[width=7cm,keepaspectratio]{./figures/CalicoVersionTwo/overview-scraps}
   }
  \subfigure[Intentional Interfaces - Intention View] {
      \label{fig:calico-version-two:overviewb}
      \includegraphics[width=7cm,keepaspectratio]{./figures/CalicoVersionTwoCluster}
   }
  \subfigure[Palette] {
      \label{fig:calico-version-two:overviewc}
      \includegraphics[width=7cm,keepaspectratio]{./figures/CalicoVersionTwo/overview-palette}
   }
  \subfigure[Highlighter] {
      \label{fig:calico-version-two:overviewd}
      \includegraphics[width=7cm,keepaspectratio]{./figures/CalicoVersionTwo/highlighter}
   }
   \caption {Calico features}
   \label{fig:calico-version-two:overview}
\end{figure}%

\begin{center}
\begin{longtable}{|p{4cm}|p{4cm}|p{4cm}|p{4cm}|}
\caption{The set of design behaviors and the features that support them}\\
\hline
\textbf{Behavior} & \textbf{Supporting Feature} & \textbf{Design Principles} & \textbf{Effect} \\
\hline
\endfirsthead
\multicolumn{4}{c}%
{\tablename\ \thetable\ -- \textit{Continued from previous page}} \\
\hline
\textbf{Behavior} & \textbf{Supporting Feature} & \textbf{Design Principles} & \textbf{Effect} \\
\hline
\endhead
\hline \multicolumn{4}{r}{\textit{Continued on next page}} \\
\endfoot
\hline
\endlastfoot
\multicolumn{4}{|l|}{Kinds of sketches software designers produce} \\
\hline
1)      They draw different kinds of diagrams	&Scraps \& connectors	&User-drawn shapes are preserved Strokes drawn inside scraps are grouped implicitly Scraps \& connectors can be hierarchically composed	&Scraps \& connectors provide a unified abstraction for informally modeling a range of notations\\
\hline
2)      They produce sketches that draw what they need, and no more	&	&	&\\
\hline
a.   They only draw what they need wrt the design at hand	&Scraps \& connectors	&Calico, at its core, acts just like a whiteboard, not dictating any content	& \multirow{2}{2cm}{Can draw just what they want and nothing more} \\
\hline
b.   They use only those notational conventions that suit drawing what they need	&Scraps \& connectors	&Scraps \& connectors do not impose any notational conventions or uses &	\\
\hline
3)      They refine and evolve their sketches over time	&	&	&\\
\hline
a.   They detail their sketches with increasing notational convention	&Compositional Notations	&Provide additional visual structure and behaviors to scraps and connectors 	&General scraps \& connectors can incrementally be refined to look like and behave as different notations\\
\hline
b.   They appropriate a sketch in one notational convention into another notational convention	&Compositional Notations	&Visual structure \& behaviors of scraps and connectors can be changed 	&Scraps and connectors looking like and adhering to one notation can be changed into the look and behavior of another\\
\hline
4)      They use impromptu notations	&Scraps \& connectors	&Any visual convention can be adopted simply by drawing similarly shaped scraps	&New notational conventions can be introduced, used as any other, and reused\\
\hline
	&Palette	&Any scrap, connector, or set of them can be stored in a palette for later reuse	&\\
\hline
	&Compositional Notations	&Elementary visual looks and behavior can be randomly composed	&\\
\hline
\multicolumn{4}{|l|}{How they use the sketches to navigate through a design problem} \\
\hline
5)      They move from one perspective to another	&Scraps \& connectors	&They can mix and match different notational conventions on a single canvas	&Different perspectives can be developed both within and across canvases; in the latter case they are explicitly related\\
\hline
	&Intentional Interfaces	&Users can explicitly request a new canvas to work on a perspective	&\\
\hline
6)      They move from one alternative to another	&Scraps \& connectors	&Different alternatives can be quickly constructed by copying and moving and otherwise manipulating scraps and connectors	&Different alternatives can be developed both within a canvas and across; in the latter case, they are explicitly related\\
\hline
	&Palette	&Different alternatives can be quickly constructed by reusing elements from and copying them differently the palette	&\\
\hline
	&Intentional Interfaces	&Users can explicitly request a new canvas to work on a different alternative	&\\
\hline
7)      They move from one level of abstraction to another	&Scraps \& connectors	&Different abstractions can be quickly constructed by copying and moving and otherwise manipulating scraps and connectors	&Different levels of abstraction can be developed both within and across canvases, in the latter case, they are explicitly related\\
\hline
	&Intentional Interfaces	&Users can explicitly request a new canvas to work on a deeper level of abstraction	&\\
\hline
8)      They perform mental simulations	&Highlighter	&Users can use the highlighter to mark up their diagrams without editing them	&The design can be gestured at without modifying it\\
\hline
9)      They juxtapose sketches	&Scraps	&Uses can move perspectives, alternatives, and abstractions next to one another by moving scraps	&The pieces of the design can better be compared without looking across the canvas\\
\hline
	&Intentional Interfaces	&Users can compare canvases by zooming in on a portion of the intentional view	&\\
\hline
10)  They review their progress	&Intentional Interfaces	&Users can step back and examine their progress and process, overall and in parts, in the intention view	&The design can be reviewed more quickly since it is already structured\\
\hline
11)  They retreat to previous ideas	&Intentional Interfaces	&Users can choose to enter one canvas in the intentional view or make a new canvas at any time	&Users can return to older versions they kept\\
\hline
\multicolumn{4}{|l|}{How they collaborate on them}\\
\hline
12)  They switch between synchronous and asynchronous work	&Intentional Interfaces	&Users can choose to enter one canvas in the intention view, or make a new canvas and work separately	&Work is synchronous while in one canvas, asynchronous while in different canvases\\
\hline
13)  They explain their sketches to each other	&Highlighter	&Users can use the highlighter to draw attention to certain parts of a canvas	&Draw attention of designers to highlighted area\\
\hline
14)  They bring their work together	&Scraps \& connectors	&Scraps can pick up sketches and drop those sketches onto other scraps, merging them	&Users can return to older versions they kept\\
\hline
	&Palette	&Designers can place sketches from different canvases into palette and later merge them into a single canvas	&\\
\hline
	&Intentional Interfaces	&Allows users to merge canvases together	&\\
\label{table:calico-version-two:designbehaviors}
\end{longtable}
\end{center}



\section{Implementation}
\label{chapter:calico-version-two:implementation}

To date, we have built four iterations of Calico. The most recent iteration of Calico provides the foundation to build and experiment with the planned features. Figure \ref{figure:calico-version-two:canvas} presents Calico as it first appears when a designer enters a canvas. Users can immediately draw or write by dragging their pen across the whiteboard. Just as on a standard whiteboard, they can make any marks they wish, anywhere, in any shape. The drawing canvas is largely clear of any obstructions, with features available on the periphery to maintain the appearance of a standard whiteboard. The side panels are mirrored on both the left and right sides, and a panel at the bottom displays status messages and has a minimal set of buttons. 

The side panels have an assortment of features to help the user while sketching. The green icons belong to the features that affect the entire set of sketches. The first five icons pertain to the grid feature, a precursor to intentional interfaces described below, and help the user in navigating the canvas. The next group of three icons allows the user to erase the canvas or alert others not to erase it. The user can undo and redo their actions. The other icons pertain to the contents drawn on the canvas. The buttons enable functionality such as pen color, stroke widths, pen modes, and special types of scraps. The pen mode toggles the input mode between regular sketching, the eraser that removes regular strokes, and the highlighter.  

\begin{figure*}[tbh]
  \centering
  \includegraphics[width=16cm,keepaspectratio]{./figures/CalicoVersionTwoCanvas}
  \caption{The Calico canvas interface, including the side panels and the drawing space.}
  \label{figure:calico-version-two:canvas}
\end{figure*}

\subsection{Scraps}

Scraps and connectors are the first major features we built. As mentioned in Section \ref{chapter:calico-version-two:features}, scraps enable users to both manipulate content and create informal diagrams as in Figure \ref{fig:calico-version-two:scrapse}. In order to support users in these activities, scraps have developed a complex set of functionality. Scraps can be created while sketching using the scrapping gestures in Figure \ref{fig:calico-version-two:scrapsa} and \ref{fig:calico-version-two:scrapsb}, which transform drawn content into the selection scraps in Figure \ref{fig:calico-version-two:scrapsc} for quick manipulation of sketches, or into the regular scraps in Figure \ref{fig:calico-version-two:scrapsd} for representing diagrams. Once made into a regular scrap, it becomes an implicit group that is universally manipulatable, stackable, and relatable, as in Figure \ref{fig:calico-version-two:scrapse}. We review these functionalities one by one.

\emph{Scrapping gestures.} Calico has two basic gestures that it uses for creating scraps directly while drawing. The first gesture creates a scrap by circumscribing an area and releasing the pen in a landing zone, as seen in Figure \ref{fig:calico-version-two:scrapsa}. In order to not interfere with the sketching activity itself, the landing zone will only appear if the stroke is sufficiently long enough. The second gesture for creating a scrap is done by pressing-and-holding the pen inside a stroke that is already circumscribing an area, which animates a dotted red circle and creates a scrap from the enclosing stroke, as in Figure \ref{fig:calico-version-two:scrapsb}. The second gesture allows the user to create a scrap after the stroke has already been created, either as a recovery mechanism if the user missed the landing zone of the first strategy, or if they choose to convert an already existing sketch into a scrap. 

\begin{figure}%
  \centering
  \subfigure[Landing zone scrapping gesture] {
      \label{fig:calico-version-two:scrapsa}
	  \includegraphics[width=7cm,keepaspectratio]{./figures/CalicoVersionTwo/scrapgestures-a}
   }
  \subfigure[Press-and-hold scrapping gesture] {
      \label{fig:calico-version-two:scrapsb}
      \includegraphics[width=7cm,keepaspectratio]{./figures/CalicoVersionTwo/scrapgestures-b}
   }
  \subfigure[Selection scrap] {
      \label{fig:calico-version-two:scrapsc}
      \includegraphics[width=7cm,keepaspectratio]{./figures/CalicoVersionTwo/scrapgestures-c}
   }
  \subfigure[Scrap] {
      \label{fig:calico-version-two:scrapsd}
      \includegraphics[width=7cm,keepaspectratio]{./figures/CalicoVersionTwo/scrapgestures-d}
   }
   \subfigure[Stacked scraps with connectors] {
      \label{fig:calico-version-two:scrapse}
      \includegraphics[width=14cm,keepaspectratio]{./figures/CalicoVersionTwo/scrapgestures-e}
   }
   \caption {Scrap functionality}
   \label{fig:calico-version-two:scraps}
\end{figure}%

\emph{Selection and regular scraps.} Scraps created from the pen gestures are first made into selection scraps (Figure \ref{fig:calico-version-two:scrapsc}), which disappear after the scrap loses focus. In past experiments of Calico, we saw that users often created scraps for the sole intention of manipulating some set of sketched content and immediately dismissed the scrap after using them for one or two actions \cite{mangano2012design}. The selection scrap streamlines this behavior. Upon creation, the selection scrap highlights any content inside of it, and allows the user to manipulate the selected contents using the bubble menu surrounding the scrap, as seen in Figure \ref{fig:calico-version-two:scrapsc}. When the selection scrap loses focus, it immediately disappears and returns its contents to the canvas. If the user wishes to permanently retain the scrap, they can tap any of the two scrap icons in the upper left of the bubble menu, which transforms it into a regular scrap, as in Figure \ref{fig:calico-version-two:scrapsd}. Selection scraps allow users to benefit from the manipulation capabilities of scraps without forcing the contents of their sketch to be scraps.

\emph{Implicit grouping.} Scraps build upon the approach taken in Translucent Patches \cite{Kramer}, which allows users to explicitly declare an area as a group. Anything that is either entirely circumscribed in the first place or otherwise written or drawn in this area afterwards is automatically part of the group. Consider the sketch of ATM in Figure \ref{fig:calico-version-two:scrapse}. It was first drawn on the canvas, then circumscribed by the stylus to become a scrap. The scrap is now a persistent object with a blue background. Any further additions to the ATM scrap, or any other scrap in Figure \ref{fig:calico-version-two:scrapse}, automatically become part of that scrap.

\emph{Manipulation.} Scraps are movable, copy-able, rotatable, and resizable. Tapping or writing on a scrap immediately highlights it and presents the bubble menu on its periphery, as in Figure \ref{fig:calico-version-two:scrapsc} and \ref{fig:calico-version-two:scrapsd}. These seemingly innocuous actions in reality represent a significant improvement over the standard whiteboard: content drawn can be rapidly reorganized. It particularly is important that such reorganization takes place in the language of the user: elements that they have deemed of sufficient importance to promote to being a scrap are the elements that are manipulated.

\emph{Stacking.} Moving a scrap to a position where it is entirely overlapped by another scrap attaches it to the scrap behind it, allowing users to quickly create a stack of scraps (thereby creating hierarchically composed groups), as one would a pile of papers. For instance, the scraps labeled ``Deposit'', ``Withdrawal'', and ``CheckBalance'' in Figure \ref{fig:calico-version-two:scrapse} are part of the scrap labeled ``Transactions''. If ``Transactions'' is moved, ``Deposit'', ``Withdrawal'', and ``CheckBalance'' are moved as well. Dragging a scrap off of another scrap un-groups it. Moving the scrap labeled ``Deposit'' from its current location will ungroup it from ``Transactions'' and, in one fluid motion, group it with ``User Interface'' when the user drops it there. Note that dragging a scrap implicitly moves it to the top of the order of scraps; scraps do not slide under other scraps.

\emph{Connectors.} By dragging the digital stylus from one scrap to another, the pen stroke becomes highlighted and presents the user with an icon to transform that stroke into a connector. The connector preserves the shape of the stroke, but is decorated with an arrowhead. The connector is persistent and anchored to the places where it originated and ended. When scraps are moved, the connectors move accordingly and keep the two scraps related. In Figure \ref{fig:calico-version-two:scrapsa}, the scrap ``ATM'' relates to the ``Transactions'' and ``User Interface'' scraps, and the scrap ``Transaction'' relates to ``BankDB''.

\subsection{Intentional interfaces}

The intentional interfaces feature presents a novel method to organize sketches across canvases. This feature is not only motivated by the second category of behaviors in Section \ref{chapter:motivation}, but also by our experiences with the grid. In both user testing and our own extensive use of the grid, we observed that people implicitly related the content between canvases. They often wanted a new canvas for a certain purpose, but did not necessarily want to go to the grid to find an empty one. Intentional interfaces make it easier to create a new empty canvas, and enable the capture of relationships among canvases. Specifically, we will add a ``new canvas'' button to replace the grid icon and interface, and use tagging to identify the intention of a particular new canvas (Figure 11). To see the resulting network of canvases, the user can zoom out to the intention view (Figure 12), where they can view a representation of all of the design sketches as they have created thus far. 

In capturing the intention of the user with tags, we can maintain a meaningful relationship between canvases. Upon entering a new canvas that was created using either the ``new'' or ``copy'' buttons (N and C in Figure 13, respectively), the user is presented with a set of tags from which they can choose an intention for that canvas (Figure 11). For example, tagging a canvas as an ``alternative'' means that the current canvas represents an alternative idea to the previous one, and similarly choosing the ``perspective'' tag means that this canvas represents another perspective to the previous one. Conversely, a new canvas that is not tagged will not maintain a link to the previous canvas. 

Given that the intentional interfaces feature is an iteration of the grid, we carry over many of the lessons learned from the grid. First, the locality of related information should be preserved. Also known as the ``neighbor knowledge awareness problem'' \cite{dekel2007notation}, canvases that are initially located adjacent to one another typically have related information, but that relation can be lost if the canvases are moved away from one another. With respect to the grid, users typically clustered related content together, but, due to fixed spatial constraints, the grid does not scale and frequently must be manually reorganized. The intention view, in contrast, does scale. Second, grouped content is segregated from other grouped content. In the grid, one person’s ``space'' of canvases can cross with another person’s space of canvases. The intention view eliminates that problem, helping both with navigation of canvases and people working together in the same design space. Further, canvases can now be labeled, providing additional support to organizing clusters. Third, users often want a new canvas without the cognitive burden of deciding where in the grid it should be placed. We streamline this process in the intentional interfaces feature by allowing the user to create a new canvas from within a canvas, and further do not burden a user to create a tag unless they explicitly want to link their canvases.
  
\subsection{Palette}

The palette is the final feature that is currently implemented, which saves a template of a scrap that can be reused (See Figure \ref{fig:calico-version-two:palettea}). Each scrap contains a palette button on the top right of its bubble menu (Figure \ref{fig:calico-version-two:paletteb}), which adds that scrap to the palette bar. Once added, the user can reuse the scrap by pressing the pen on the image of the icon that they wish to use, and dragging it back onto the canvas. The items on the palette bar are globally available both across all canvases in the grid, and all users connected to the same server. Further, the palette has multiple sets that the user can switch between using the up and down arrows on the palette bar in Figure \ref{fig:calico-version-two:palettea}, or can also create a new set by pressing the plus icon. The palette bar contains several other features for managing the palette, such as saving a palette to disk, opening a saved palette, deleting a palette, and importing images into a palette. When the user is finished using the palette, they can toggle it to be invisible from the bottom menu bar, as seen in Figure \ref{figure:calico-version-two:canvas}.

\begin{figure}
  \centering
  \subfigure[Palette bar with scraps] {
      \label{fig:calico-version-two:palettea}
      \includegraphics[width=8.5cm,keepaspectratio]{./figures/CalicoVersionTwo/palette-a}
   }
  \subfigure[Palette icon on a scrap’s bubble menu] {
      \label{fig:calico-version-two:paletteb}
      \includegraphics[width=6.5cm,keepaspectratio]{./figures/CalicoVersionTwo/palette-b}
   }
   \caption {Scraps can be added to palette for rapid reuse}
   \label{fig:calico-version-two:palette}
\end{figure}

\subsection{Highlighter}

The highlighter separates the act of highlighting from drawing. This functionality allows the designer to point, circle, and draw sequences over diagrams, but the strokes made by the designer fade a few seconds after they are made. While a small piece of functionality, the highlighter serves an important role. It allows a group of designers, and even the individual designer, to perform mental simulations over a sketch to explain them to each other, both while at the same whiteboard and over the network in distributed cases

\begin{figure*}[tbh]
  \centering
  \includegraphics[width=9cm,keepaspectratio]{./figures/CalicoVersionTwo/highlighter}
  \caption{Highlighter}
  \label{figure:calico-version-two:highlighter}
\end{figure*}

%%% Local Variables: ***
%%% mode: latex ***
%%% TeX-master: "thesis.tex" ***
%%% End: ***
