\chapter{Calico Version Two}
\label{chapter:calico-version-two}

Equipped with the lessons from Chapters \ref{chapter:calico-version-one} and \ref{chapter:notation-paper}, I have gone through an iterative process of designing, implementing, and evaluating various possible incarnations of Calico. Through experiments and trials with the technology over several years, including using the tool in my own meetings, deploying it in the classroom \cite{Loksa2013}, and formal experimentation \cite{mangano2012design}, I have slowly, but surely, developed an understanding of how to mesh more advanced functionality with the fluidity that the whiteboard experience demands. I have specifically refined what first was a broad range of exploratory features into a small, cohesive set of just five features that form the conceptual core of Calico.

The laboratory evaluations in Chapter \ref{chapter:calico-version-one} have given insight into the usefulness and shortcomings of the features of Calico Version One, such as scraps, the palette, and the grid. Participants found scraps useful to manipulate content, but found them difficult to use as representations for class diagrams because scraps could not be resized or rotated. Similarly, participants found the palette useful as a global clipboard to transfer content across canvases, but the inflexible nature of scraps made reusing them from the palette difficult. The grid itself proved to be a highly used feature in navigating and organizing sketches, but participants reported wanting more organizational power such as grouping sets of canvases into topics.

The design sessions with professional software designers in Chapter \ref{chapter:notation-paper} revealed greater insight into the design activities that were not captured in Chapter \ref{chapter:calico-version-one}. Regarding the representations professional designers used, lists were ubiquitous in supporting several aspects of design, including the user interface, software architecture, and requirements. The professional designers also demonstrated several methods of refining and evolving their sketches, including refining them in place by boxing items in lists, or redrawing sketches in place. Further, the designers often did not edit their sketches during activities such as mental simulations or reviewing requirements, but instead simply pointed at sketches the majority of the time while they talked.

From the insights found in Chapter \ref{chapter:calico-version-one} and Chapter \ref{chapter:notation-paper}, I have arrived at the final five features in this chapter, which I believe to be effective in supporting the fourteen design behaviors identified in Chapter \ref{chapter:motivation}. For reference, Table \ref{table:calico-version-two:designbehaviors} summarizes the design behaviors here in this chapter, for reference. The features in this chapter represent the final iteration that will be evaluated in Chapter \ref{chapter:evaluation}.

In this chapter, I first describe the motivation that lead to the changes I made to each of the Calico Version One features and to the introduction of new ones. The motivation is explained in terms of the design behaviors and the final feature set is mapped to their motivating design behaviors. Next, I describe the features in detail, eliciting the exact functionality for each feature. Lastly, I describe the implementation details of the system.

\section{Updated features}
\label{chapter:calico-version-two:features}

In this section, I review the changes between Calico Version One and Calico Version Two, as well as the motivations that lead to these changes. I do not go into great detail for the features here, but only refer to the features at a high level. Section \ref{chapter:calico-version-two:implementation} describes the features in detail.

\subsection{Changes to basic sketching}

With respect to basic sketching, Calico Version Two includes added functionality that enriches the basic sketching experience of Calico Version One. The first fundamental change to Calico is its transition from a single client application to a distributed, synchronous sketching application. One of the greatest challenges encountered by participants in Chapter \ref{chapter:calico-version-one} was the inability to perform asynchronous work. Given the constraints of the technology involved with the Hitachi Starboard FX-DUO, we chose to reimplement Calico as a distributed application so that users on separate machines could sketch at the same time. The inclusion of distributed, synchronous sketching better supports design behavior 12, in which designers switch between synchronous and asynchronous work. Several other small features were included as well, such as the ability to import images and email canvases, which help to bring outside work into Calico.

\subsection{Changes to scraps}

With respect to scraps, much of the interaction changed between Calico Version One and Calico Version Two based on our feedback in Chapter \ref{chapter:calico-version-one}. First is the introduction of selection scraps (center of Figure \ref{fig:calico-version-two:overviewa}). In Chapter \ref{chapter:calico-version-one}, participants in the user evaluations often used scraps for single actions, such as moving, and then would ``drop'' the scrap after moving its content. Based on this usage, the behavior of scraps was changed such that newly created scraps instead function as a lasso, and will disappear unless they are ``pinned'' by the user, as in Figure \ref{fig:calico-version-two:overviewa}. This behavior, termed selection scrap, increases the fluidity of using scraps by reducing the amount of actions needed to perform common actions such as moving sketches. 

Second is the functionality to transform existing sketches into scraps. Participants in the user trials of Chapter \ref{chapter:calico-version-one} reported that they sometimes did not use scraps because it required them to plan ahead. That is, they did not know early on some areas would need to be an object later. Retroactively ``promoting'' such an area to a scrap lead to objects with double boundaries. To solve this, this second behavior addresses this issue and further supports refinement of sketches (design behavior 3) by allowing the user to create scraps when they need them without as much need to plan ahead. 

Third, the menu for scraps was revised to appear as a ``bubble menu'' (Figure \ref{fig:calico-version-two:overviewa}), which presents the buttons on the periphery of the scrap. Users of Calico Version One head difficulty ``hitting'' the menu button, which was small, and also stated that the button and menu contents occluded the visibility of the scraps themselves. Items in the bubble menu are much easier to tap on and do not occlude associated scrap contents. Furthermore, it was much easier to add new previously requested features such as resize and rotate, since the bubble menu has room to accommodate additional items. The bubble menu further replaces much of the functionality enabled by the gesture system in Calico Version One. Finally, in exit interviews for the laboratory evaluation of Chapter \ref{chapter:calico-version-one}, participants reported that they did not use a number of features, because they were not discoverable through the set of gestures. The bubble menu instead uses graphical icons to offer this functionality. Gestures have largely been removed from Calico Version Two, with the exception of the press-and-hold gesture.

Fourth, connectors were revised such that they could also be created from existing sketches, matching the revised behavior of scraps, and also now retain their shape (as shown in Figure \ref{fig:calico-version-two:overviewa}). Preserving their shape provides the user with greater control in how they express their arrow, improving their ability to draw different types of connectors (design behavior 1). 

\begin{figure*}[tbh]
  \centering
  \includegraphics[width=16cm,keepaspectratio]{./figures/CalicoVersionTwo/overview-scraps}
  \caption{Regular sketches can be transformed into selection scraps, which can further be made permanent by ``pinning'' the scrap.}
  \label{fig:calico-version-two:overviewa}
\end{figure*}

Additionally, specialized scraps, such as text-scraps and list-scraps, were introduced. Chapter \ref{chapter:notation-paper} demonstrated the importance of creating and managing lists of text alongside diagrams. Professional designers created, referred to, and checked off items in lists throughout their design sessions. Participants in the laboratory evaluation from Chapter \ref{chapter:calico-version-one} also devoted entire canvases to writing out requirements as plain text. Text-scraps and list-scraps target these actions specifically by automatically organizing scraps for the user.

Despite these major chances to the interactions of scraps, much of their fundamental nature remains the same in order to continue supporting the same design behaviors. Scraps still retain the shape of the pen stroke used to create them. Parenting is enabled by stacking scraps on top of one another to implicitly group them. Both of these functionalities contribute towards supporting the first and second design behaviors.

\subsection{Changes to palette}

With respect to the palette, its basic operation has changed little. The palette continues to serve its purpose as a global source to save scraps, which can immediately be reused anywhere by dragging it back to the canvas from the palette. However, the palette does include updates that allow the user to create multiple palettes of scraps that they can move between or delete. 

The palette has, however, been greatly enriched by the other new features. The revised distributed nature of Calico allows the items in the palette to be automatically shared by everyone who is connected to the same server. The ability to resize and rotate scraps, as well as the functionality to save scraps with imported images into the palette, increase the ability to create and reuse scraps pertaining to a project. They, in particular, increase the palette's ability to support the creation of impromtu notations (design behavior 4).

\subsection{Removal of grid and introduction of intentional interfaces}

With respect to the grid, the feature has been replaced by intentional interfaces. While the grid has proven useful in past evaluations, intentional interfaces not only enable the user to organize their canvases like the grid, but also provide the user with greater ability than with the grid. Here I review intentional interfaces, as well as the shortcomings of the grid that intentional interfaces address. 

\begin{figure*}[tbh]
  \centering
  \includegraphics[width=16cm,keepaspectratio]{./figures/CalicoVersionTwoCluster}
  \caption{Intentional Interfaces - Intention View.}
  \label{fig:calico-version-two:overviewb}
\end{figure*}

Intentional interfaces is composed of three pieces. First, rather than a grid layout, intentional interfaces organizes canvases into a radial layout (Figure \ref{fig:calico-version-two:overviewb}), which can further be organized by branching. This functionality was created to help provide a greater grouping mechanism for canvases. While the grid did serve as an effective and straightforward mechanism to organize canvases within a design session, in practice, the grid has several shortcomings that merited a more complex feature such as intentional interfaces. For example, the participants in the laboratory evaluation in Chapter \ref{chapter:calico-version-one} grouped canvases of similar topics, such as requirements of interface mockups, in close proximity to one another. In exit interviews, participants expressed that they desired to go one step further with explicit grouping. 

\begin{figure*}[tbh]
  \centering
  \includegraphics[width=16cm,keepaspectratio]{./figures/CalicoVersionTwo/overviewb-tags}
  \caption{Canvases are related to one another within intentional interfaces using tags.}
  \label{fig:calico-version-two:overviewb-tags}
\end{figure*}

In the second piece of intentional interfaces, each canvas has a tag that explicitly associates it with the previous canvas it branches out from, i.e., it is an ``alternative'', ``level of abstraction'', and so on, of the previous canvas. Figure \ref{fig:calico-version-two:overviewb-tags} depicts two canvases associated in this manner, in which one canvas is marked as an ``alternative'' of another. In Chapter \ref{chapter:calico-version-one}, the participants were observed to shift between perspectives of the same entity. In Chapter \ref{chapter:notation-paper}, the professional designers demonstrated several explicit reasons for navigating between canvases, such as shifting between perspectives, abstractions, or alternatives. Tags enable the user to declare their intention when creating a canvas, thus providing a lightweight mechanism for supporting many of the navigation design behaviors 5, 6, and 7 in Section \ref{chapter:motivation:navigation}. 

The third piece is the breadcrumb bar (top-left of Figure \ref{fig:calico-version-two:overviewb}), which allows designers to navigate through the canvases. This third feature enriches the radial layout with an additional method to navigate between the canvases, and displays the associated tags as well.

\subsection{Introduction of fading highlighter}

In addition to the previous features, a new feature, the fading highlighter, is introduced to support the explanation of sketches (design behavior 13). The highlighter separates the act of highlighting from drawing. This functionality allows the designer to point, circle, and draw sequences over diagrams, but the strokes made by the designer fade a few seconds after they are made. While a small piece of functionality, the highlighter serves an important role. It allows a group of designers, and even the individual designer, to perform mental simulations (design behavior 8) and reviews of progress (design behavior 9), both while at the same whiteboard and over the network in distributed cases. The professional designers in Chapter \ref{chapter:notation-paper} demonstrated that in nearly 80 percent of reasoning activities, they do not manipulate the sketch itself, but rather only look at or gesture over a sketch. The highlighter enables users to accomplish the same purpose of gesturing over diagrams without modifying their sketches, yet be much more explicit in expressing their intent.

\begin{figure*}[tbh]
  \centering
  \includegraphics[width=8cm,keepaspectratio]{./figures/CalicoVersionTwo/highlighter}
  \caption{Fading highlighter.}
  \label{fig:calico-version-two:overviewd}
\end{figure*}

%\begin{figure}%
%  \centering
%  \subfigure[Scraps] {
%      \label{fig:calico-version-two:overviewa}
%	  \includegraphics[width=7cm,keepaspectratio]{./figures/CalicoVersionTwo/overview-scraps}
%   }
%  \subfigure[Intentional Interfaces - Intention View] {
%      \label{fig:calico-version-two:overviewb}
%      \includegraphics[width=7cm,keepaspectratio]{./figures/CalicoVersionTwoCluster}
%   }
%  \subfigure[Palette] {
%      \label{fig:calico-version-two:overviewc}
%      \includegraphics[width=7cm,keepaspectratio]{./figures/CalicoVersionTwo/overview-palette}
%   }
%  \subfigure[Highlighter] {
%      \label{fig:calico-version-two:overviewd}
%      \includegraphics[width=7cm,keepaspectratio]{./figures/CalicoVersionTwo/highlighter}
%   }
%   \caption {Calico features}
%   \label{fig:calico-version-two:overview}
%\end{figure}%

\subsection{Features and design behaviors they support}

With the changes and additional features introduced in this section, Calico now supports all fourteen design behaviors. Table \ref{table:calico-version-two:designbehaviors} provides a mapping between the design behaviors and the features that support them, as well as the reason why I believe the features support the respective design behavior.

\begin{center}
\begin{longtable}{|p{4cm}|p{4cm}|p{4cm}|p{4cm}|}
\caption{The set of design behaviors and the features that support them}\\
\hline
\textbf{Design Behavior} & \textbf{Supporting Feature} & \textbf{Design Principles} & \textbf{Reason} \\
\hline
\endfirsthead
\multicolumn{4}{c}%
{\tablename\ \thetable\ -- \textit{Continued from previous page}} \\
\hline
\textbf{Behavior} & \textbf{Supporting Feature} & \textbf{Design Principles} & \textbf{Reason} \\
\hline
\endhead
\hline \multicolumn{4}{r}{\textit{Continued on next page}} \\
\endfoot
\hline
\endlastfoot
\multicolumn{4}{|c|}{\textbf{Kinds of sketches software designers produce}} \\
\hline
1. They draw different kinds of diagrams	&Scraps \& connectors	&User-drawn shapes are preserved; strokes drawn inside scraps are grouped implicitly; scraps \& connectors can be hierarchically composed	&Scraps \& connectors provide a unified abstraction for informally modeling a range of notations\\
\hline
2. They produce sketches that draw what they need, and no more	&	&	&\\
\cline{1-4}
2a. They only draw what they need w.r.t. the design at hand	&Scraps \& connectors	&Calico, at its core, acts just like a whiteboard, not dictating any content	& %\multirow{2}{4cm}{
Can draw just what they want and nothing more
%} 
\\
\cline{1-4}
2b. They use only those notational conventions that suit drawing what they need	&Scraps \& connectors	&Scraps \& connectors do not impose any notational conventions or uses &	Can draw just what they want and nothing more\\
\hline
3.      They refine and evolve their sketches over time	&	&	&\\
\hline
3a.   They detail their sketches with increasing notational convention	&Scraps \& connectors&Provide additional visual structure and behaviors to scraps and connectors 	&General scraps \& connectors can incrementally be refined to look like and behave as different notations\\
\cline{1-4}
3b.   They appropriate a sketch in one notational convention into another notational convention	&Scraps \& connectors&Visual structure \& behaviors of scraps and connectors can be changed 	&Scraps and connectors looking like and adhering to one notation can be changed into the look and behavior of another\\
\cline{1-4}
4.      They use impromptu notations	&Scraps \& connectors	&Any visual convention can be adopted simply by drawing similarly shaped scraps	& New notational conventions can be introduced, used as any other, and reused\\
\cline{2-3}
	&Palette	&Any scrap, connector, or set of them can be stored in a palette for later reuse	&\\
\hline
%	&Scraps and connectors &Elementary visual looks and behavior can be randomly composed	&\\
%\hline
\multicolumn{4}{|c|}{\textbf{How they use the sketches to navigate through a design problem}} \\
\hline
5.      They move from one perspective to another	&Scraps \& connectors	&They can mix and match different notational conventions on a single canvas	&Different perspectives can be developed within a canvas	\\
\cline{2-4}
	&Intentional Interfaces	&Users can explicitly request a new canvas to work on a perspective	& Canvases with different perspectives are explicitly related\\
\hline
6.      They move from one alternative to another	&Scraps \& connectors	&Different alternatives can be quickly constructed by copying and moving and otherwise manipulating scraps and connectors	&Different alternatives can be developed within a canvas\\
\cline{2-3}
	&Palette	&Different alternatives can be quickly constructed by reusing elements from the palette and composing them differently &\\
\cline{2-4}
	&Intentional Interfaces	&Users can explicitly request a new canvas to work on a different alternative	&Canvases with different alternatives are explicitly related\\
\hline
7.      They move from one level of abstraction to another	&Scraps \& connectors	&Different abstractions can be quickly constructed by copying and moving and otherwise manipulating scraps and connectors	&Different levels of abstraction can be developed within a canvas\\
\cline{2-4}
	&Intentional Interfaces	&Users can explicitly request a new canvas to work on a deeper level of abstraction	&Canvases with different abstractions are explicitly related\\
\cline{1-4}
8.      They perform mental simulations	&Highlighter	&Users can use the highlighter to mark up their diagrams without editing them	&The design can be gestured at without modifying it\\
\hline
9.      They juxtapose sketches	&Scraps	&Uses can move perspectives, alternatives, and abstractions next to one another by moving scraps	&The pieces of the design can better be compared without looking across the canvas\\
\cline{2-4}
	&Intentional Interfaces	&Users can compare canvases by zooming in on a portion of the intentional view	&Content across canvases can be more easily juxtaposed without manipulating content within canvases\\
\hline
10.  They review their progress	&Intentional Interfaces	&Users can step back and examine their progress and process, overall and in parts, in the intention view	&The tags and visual structure provide some insight into the rationale of canvases\\
\hline
11.  They retreat to previous ideas	&Intentional Interfaces	&Users can choose to enter one canvas in the intentional view or make a new canvas at any time	&Users can return to older versions they kept\\
\hline
\multicolumn{4}{|c|}{\textbf{How they collaborate on them}}\\
\hline
12.  They switch between synchronous and asynchronous work	&Intentional Interfaces	&Users can choose to enter one canvas in the intention view, or make a new canvas and work separately	&Work is synchronous while in one canvas, asynchronous while in different canvases\\
\hline
13.  They explain their sketches to each other	&Highlighter	&Users can use the highlighter to draw attention to certain parts of a canvas	&Draw attention of designers to highlighted area\\
\hline
14.  They bring their work together	&Palette	&Designers can place sketches from different canvases into palette and later merge them into a single canvas		&Users can return to older versions they kept\\
\cline{2-3}
	&Intentional interfaces	&Users can navigate back and forth between canvases, and generate new ones&
\label{table:calico-version-two:designbehaviors}
\end{longtable}
\end{center}



\section{Implementation}
\label{chapter:calico-version-two:implementation}
    
In the previous section, I motivated the changes to existing features and the introduction of new features for Calico Version Two. In this section, I explain the \emph{how} for each feature, explaining how each feature works in detail.

\subsection{Implementation of basic sketching}

Figure \ref{figure:calico-version-two:canvas} presents Calico as it first appears when a designer enters a canvas. Users can immediately draw or write by dragging their pen across the whiteboard. Just as on a standard whiteboard, they can make any marks they wish, anywhere, in any shape. The drawing canvas is largely clear of any obstructions, with features available on the periphery to maintain the appearance of a standard whiteboard. The side panels are mirrored on both the left and right sides, and a panel at the bottom displays status messages and has a minimal set of auxiliary buttons. 

\begin{figure*}[tbh]
  \centering
  \includegraphics[width=16cm,keepaspectratio]{./figures/CalicoVersionTwo/CalicoVersionTwoCanvas}
  \caption{The Calico canvas interface, including the side panels and the drawing space.}
  \label{figure:calico-version-two:canvas}
\end{figure*}

The side panels have an assortment of features to help the user while sketching. The green icons belong to the features that affect the entire set of sketches. The first five icons pertain to intentional interfaces described below, and help the user in creating and navigating canvases. The next icon allows the user to erase the canvas. The user can undo and redo their actions. The other icons pertain to the contents drawn on the canvas. The buttons enable functionality such as pen color, stroke widths, pen modes, and special types of scraps. The pen mode toggles the input mode between regular sketching, the eraser that removes regular strokes, and the highlighter.  

\subsection{Implementation of scraps}

As mentioned in Section \ref{chapter:calico-version-two:features}, scraps enable users to both manipulate content and create informal diagrams as in Figure \ref{fig:calico-version-two:scrapse}. In order to support users in these activities, scraps have developed a unique set of functionality. Scraps can be created while sketching using the scrapping gestures in Figure \ref{fig:calico-version-two:scrapsa} and \ref{fig:calico-version-two:scrapsb}, which transform drawn content into the selection scraps in Figure \ref{fig:calico-version-two:scrapsc} for quick manipulation of sketches, or into the regular scraps in Figure \ref{fig:calico-version-two:scrapsd} for representing diagrams. Once made into a regular scrap, it becomes an implicit group that is universally manipulatable, stackable, and relatable, as in Figure \ref{fig:calico-version-two:scrapse}. We review these functionalities one by one.

\subsubsection{Scrapping gestures} Calico has two basic gestures that it uses for creating scraps directly while drawing. The first gesture creates a scrap by circumscribing an area and releasing the pen in a landing zone, as seen in Figure \ref{fig:calico-version-two:scrapsa}. In order to not interfere with the sketching activity itself, the landing zone will only appear if the stroke is sufficiently long enough. The second gesture for creating a scrap is done by pressing-and-holding the pen inside a stroke that is already circumscribing an area, which animates a dotted red circle and creates a scrap from the enclosing stroke, as in Figure \ref{fig:calico-version-two:scrapsb}. The second gesture allows the user to create a scrap after the stroke has already been created, either as a recovery mechanism if the user missed the landing zone of the first strategy, or if they choose to convert an already existing sketch into a scrap. The ability to create scraps from plain sketches helps support the refinement of sketches (design behavior 3).

\begin{figure}%
  \centering
  \subfigure[Landing zone scrapping gesture] {
      \label{fig:calico-version-two:scrapsa}
	  \includegraphics[width=7cm,keepaspectratio]{./figures/CalicoVersionTwo/scrapgestures-a}
   }
  \subfigure[Press-and-hold scrapping gesture] {
      \label{fig:calico-version-two:scrapsb}
      \includegraphics[width=7cm,keepaspectratio]{./figures/CalicoVersionTwo/scrapgestures-b}
   }
  \subfigure[Selection scrap] {
      \label{fig:calico-version-two:scrapsc}
      \includegraphics[width=5cm,keepaspectratio]{./figures/CalicoVersionTwo/scrapgestures-c}
   }
  \subfigure[Scrap] {
      \label{fig:calico-version-two:scrapsd}
      \includegraphics[width=5cm,keepaspectratio]{./figures/CalicoVersionTwo/scrapgestures-d}
   }
   \subfigure[Stacked scraps with connectors] {
      \label{fig:calico-version-two:scrapse}
      \includegraphics[width=14cm,keepaspectratio]{./figures/CalicoVersionTwo/scrapgestures-e}
   }
   \caption {Scrap functionality}
   \label{fig:calico-version-two:scraps}
\end{figure}%

\subsubsection{Selection and regular scraps} Scraps created from the pen gestures are first made into selection scraps (Figure \ref{fig:calico-version-two:scrapsc}), which disappear after the scrap loses focus. Upon creation, the selection scrap highlights any content inside of it, and allows the user to manipulate the selected contents using the bubble menu surrounding the scrap, as seen in Figure \ref{fig:calico-version-two:scrapsc}. When the selection scrap loses focus, it immediately disappears and returns its contents to the canvas. If the user wishes to permanently retain the scrap, they can tap any of the two scrap icons in the upper left of the bubble menu, which transforms it into a regular scrap, as in Figure \ref{fig:calico-version-two:scrapsd}. Selection scraps allow users to benefit from the manipulation capabilities of scraps without forcing the contents of their sketch to be scraps.

\subsubsection{Implicit grouping} Scraps build upon the approach taken in Translucent Patches \cite{Kramer}, which allows users to explicitly declare an area as a group. Anything that is either entirely circumscribed in the first place or otherwise written or drawn in this area afterwards is automatically part of the group. Consider the sketch of ATM in Figure \ref{fig:calico-version-two:scrapse}. It was first drawn on the canvas, then circumscribed by the stylus to become a scrap. The scrap is now a persistent object with a blue background. Any further additions to the ATM scrap, or any other scrap in Figure \ref{fig:calico-version-two:scrapse}, automatically become part of that scrap.

\subsubsection{Manipulation} Scraps are movable, copy-able, rotatable, and resizable. Tapping or writing on a scrap immediately highlights it and presents the bubble menu on its periphery, as in Figure \ref{fig:calico-version-two:scrapsc} and \ref{fig:calico-version-two:scrapsd}. These seemingly innocuous actions in reality represent a significant improvement over the standard whiteboard: content drawn can be rapidly reorganized. It particularly is important that such reorganization takes place in the language of the user: elements that they have deemed of sufficient importance to promote to being a scrap are the elements that are manipulated.

\subsubsection{Stacking} Moving a scrap to a position where it is entirely overlapped by another scrap attaches it to the scrap behind it, allowing users to quickly create a stack of scraps (thereby creating hierarchically composed groups), as one would a pile of papers. For instance, the scraps labeled ``Deposit'', ``Withdrawal'', and ``CheckBalance'' in Figure \ref{fig:calico-version-two:scrapse} are part of the scrap labeled ``Transactions''. If ``Transactions'' is moved, ``Deposit'', ``Withdrawal'', and ``CheckBalance'' are moved as well. Dragging a scrap off of another scrap un-groups it. Moving the scrap labeled ``Deposit'' from its current location will ungroup it from ``Transactions'' and, in one fluid motion, group it with ``User Interface'' when the user drops it there. Note that dragging a scrap implicitly moves it to the top of the order of scraps; scraps do not slide under other scraps.

\subsubsection{Connectors} By dragging the digital stylus from one scrap to another, the pen stroke becomes highlighted and presents the user with an icon to transform that stroke into a connector. Alternative, as with scraps, connectors can be created retroactively by press-and-holding a stroke that begins and ends in a scrap, and then tapping the ``create connector'' button. The connector preserves the shape of the stroke, but is decorated with an arrowhead. The connector is persistent and anchored to the places where it originated and ended. When scraps are moved, the connectors move accordingly and keep the two scraps related. In Figure \ref{fig:calico-version-two:scrapse}, the scrap ``ATM'' relates to the ``Transactions'' and ``User Interface'' scraps, and the scrap ``Transaction'' relates to ``BankDB''.

\subsubsection{List-scraps and text-scraps} List-scraps, depicted in Figure \ref{figure:calico-version-two:list-scraps}, are a specialized feature that organize scraps into a linear vertical list, and changes its boundaries to fully contain the children within it. If a regular scrap has children, the user may promote the regular scrap to a list-scrap. Promoting a scrap to a list-scrap reorganizes the items contained in it to be a list. Just as with implicit grouping of regular scraps, the user can drag scraps on top of the list to automatically group them. The list reorganizes items after inserting the recently added scrap into the vertical list. Each item in a list has an associated box, which can be checked and unchecked. Further, as with regular scraps, lists can be nested within other lists to create multi-level structures.

\begin{figure*}[tbh]
  \centering
  \includegraphics[width=8cm,keepaspectratio]{./figures/CalicoVersionTwo/list-scraps}
  \caption{List-scraps automatically organize scraps into a vertical list.}
  \label{figure:calico-version-two:list-scraps}
\end{figure*}

Text-scraps were designed to work in conjuction with list-scraps to help the user create content more quickly from the keyboard. Using either the text-scrap button on the menubar (Figure \ref{figure:calico-version-two:canvas}), or by pressing the enter key while in the canvas, the user can create a new scrap using their keyboard. If a list-scrap is already selected, the new text-scrap will be automatically appended to the end of the list.

\subsection{Implementation of palette}

The palette saves a template of a scrap that can be reused (See Figure \ref{fig:calico-version-two:palettea}). Each scrap contains a palette button on the top right of its bubble menu (Figure \ref{fig:calico-version-two:paletteb}), which adds that scrap to the palette bar. Once added, the user can reuse the scrap by pressing the pen on the image of the icon that they wish to use, and dragging it back onto the canvas. The items on the palette bar are globally available both across all canvases in the grid, and all users connected to the same server. Further, the palette has multiple sets that the user can switch between using the up and down arrows on the palette bar in Figure \ref{fig:calico-version-two:palettea}, or can also create a new set by pressing the plus icon. This enables users to manage multiple related sets of symbols. The palette bar contains several other features for managing the palette, such as creating a new empty palette, deleting a palette, and importing images into a palette. When the user is finished using the palette, they can toggle it to be invisible from the bottom menu bar, as seen in the bottom right of Figure \ref{figure:calico-version-two:canvas}.

\begin{figure}
  \centering
  \subfigure[Palette bar with scraps] {
      \label{fig:calico-version-two:palettea}
      \includegraphics[width=8.5cm,keepaspectratio]{./figures/CalicoVersionTwo/palette-a}
   }
  \subfigure[Palette icon on a scrap’s bubble menu] {
      \label{fig:calico-version-two:paletteb}
      \includegraphics[width=4.5cm,keepaspectratio]{./figures/CalicoVersionTwo/palette-b}
   }
   \caption {Scraps can be added to palette for rapid reuse}
   \label{fig:calico-version-two:palette}
\end{figure}

\subsection{Implementation of intentional interfaces}

The intentional interfaces feature presents a novel method to organizing sketches across canvases. I break the intentional interfaces feature down into the following components: the wall that organizes topics, clusters that organize canvases into a radial pattern, tags that related canvases and organize them into branches, the canvas interaction with tags, and the breadcrumb bar that aids in the navigation between canvases.

\subsubsection{The wall perspective} 

The intentional interfaces feature is organized into three levels: the wall, clusters, and canvases. The wall provides a high level mechanism for partitioning canvases.  While the grid is no longer a feature within Calico, the grid can be thought of as being equivalent to clusters, and the wall as one level of organization higher. The inclusion of the wall is motivated by long-term, multi-month usage of Calico, in which multiple people were using the same Calico grid to manage their projects. For example, Figure \ref{figure:calico-version-two:ii-wall} depicts the wall belonging to the SDCL group, which contains content that is several years old. The wall and clusters provide a convenient way to separate projects between people. In other long term usages of Calico, users that shared a Calico grid requested a separate space to work with different topics. In the wall, grouped content is segregated from other grouped content. In the grid, one person's ``space'' of canvases can cross with another person’s space of canvases. Intention interfaces eliminates that problem, helping both with navigation of canvases and people working together in the same design space. Further, canvases can now be labeled, providing additional support to organizing clusters. 

\begin{figure*}[tbh]
  \centering
  \includegraphics[width=16cm,keepaspectratio]{./figures/CalicoVersionTwo/ii-wall}
  \caption{The Calico wall belonging to the SDCL group, which contains sketches that are several years old and are used by several group members regularly.}
  \label{figure:calico-version-two:ii-wall}
\end{figure*}

From the wall, users may navigate and manage their canvases. The user may enter a cluster by tapping on the white space within a cluster, or directly into a canvas by tapping on a canvas. They may also move canvases between clusters by dragging a canvas onto a different cluster. Clusters are created and deleted automatically, where a new cluster is created when there are no more empty clusters, and extra clusters are deleted such that there remains only one empty cluster at a time.

\subsubsection{The cluster perspective}

Within a cluster, canvases are organized into a radial layout. The radial layout provides a method to organize a scalable number of canvases, where the size of the cluster is increased to accommodate more canvases. Figure \ref{figure:calico-version-two:ii-cluster} depicts the cluster perspective of ``Nick M.'s stuff'', which is the first column, second row cluster in Figure \ref{figure:calico-version-two:ii-wall}. New canvases are added to a cluster by tapping the ``new canvas'' button in the top right of Figure \ref{figure:calico-version-two:ii-cluster}. In order to help the user in mentally organizing the set of canvases in a cluster, the organization of clusters takes its inspiration from a mechanical clock. The canvases are sorted such that their order ascends in a clockwise pattern, where the first canvas is always located at the midnight position of a mechanical clock-hand, and the most recently added canvas will always appear at the end of the clock-wise radial loop, just before the midnight clock-hand position. 

\begin{figure*}[tbh]
  \centering
  \includegraphics[width=16cm,keepaspectratio]{./figures/CalicoVersionTwo/ii-cluster}
  \caption{The cluster perspective of ``Nick M.'s stuff''.}
  \label{figure:calico-version-two:ii-cluster}
\end{figure*}

Within a cluster, the user may move canvases freely. In preliminary feedback of Calico Version Two, users reported that the cluster provided them a meta-design space, and organizing the positioning of the canvases was part of their design process. In order to accomodate this action, canvases that are moved from the radial layout become ``pinned'', and will maintain their spatial position in the cluster regardless of other canvases being added or deleted from the radial layout. Canvases may further be ``unpinned'', in which case they return to the radial layout and their positioning becomes automatically managed again.

\subsubsection{Tagging} 

Canvases in the radial layout may branch outwards from the center with the use of tagging, as depicted in Figure \ref{figure:calico-version-two:ii-cluster-branch}. In the laboratory evaluations of Chapter \ref{chapter:calico-version-one}, the users often moved to a new canvas for a particular purpose, such as exploring a new perspective, diving into more detail, or exploring an alternative. Intentional interface tags formally capture these links between canvases. Upon visiting a new canvas, the user is asked to tag the new canvas with an intention (Figure \ref{figure:calico-version-two:ii-canvas}). After choosing a tag, the new canvas is linked to the previous canvas in the radial layout. The tagged canvases form a linked list in the radial layout, thereby preserving their relationships with one another. For example, Figure \ref{figure:calico-version-two:ii-cluster-branch} depicts a diagram that is refined over several canvases using tags. The canvas in the center of the circle, ``9. MiSE - Figure 2 original'', contains the original diagram that was imported as an image. The work from the center canvas is extended into ``9.1 MiSE - Figure 2 revised'', which is tagged as a ``Continuation'' of the previous canvas. The diagram is refined and revised until the final diagram was reach in canvas ``9.4 MiSE Figure 2c''.

\begin{figure*}[tbh]
  \centering
  \includegraphics[width=16cm,keepaspectratio]{./figures/CalicoVersionTwo/ii-cluster-branch}
  \caption{A diagram is refined and revised across several canvases, which are linked with ``Continuation'' and ``Alternative'' tags.}
  \label{figure:calico-version-two:ii-cluster-branch}
\end{figure*}


Tagging is introduced in order to address the ``neighbor knowledge awareness problem'' \cite{dekel2007notation}, which was observed from participants usage of the grid in Chapter \ref{chapter:calico-version-one}. In the neighbor knowledge problem, canvases that are initially located adjacent to one another typically have related information, but that relation can be lost if the canvases are moved away from one another. With respect to the grid, users typically cluster related content together, but, due to fixed spatial constraints, the grid does not scale and frequently must be manually reorganized. The radial layout in the cluster perspective, in contrast, does scale, and preserves the relationships between canvases. 

The tag panel as presented in the canvas perspective is depicted in Figure \ref{figure:calico-version-two:ii-canvas}. The tag panel is initially populated with a set of tags inspired by design behaviors 5, 6, and 7, such as ``alternative'', ``perspective'', and ``abstraction''. The user, however, may add, edit, or delete the set of tags. Tags can be selected by tapping on the tag itself, after which the tag panel is dismissed. The user may toggle the tag panel at a later point by pressing the tag button in the status bar (bottom right of Figure \ref{figure:calico-version-two:ii-canvas}).

\begin{figure*}[tbh]
  \centering
  \includegraphics[width=16cm,keepaspectratio]{./figures/CalicoVersionTwo/ii-canvas}
  \caption{The tag panel appears in the upper-right upon entering a newly created canvas.}
  \label{figure:calico-version-two:ii-canvas}
\end{figure*}

\subsubsection{Navigating canvases and breadcrumb bar} 

From the canvas perspective, the user can create, manage, and navigate their canvases within the context of intentional interfaces. In past evaluations of Calico, users often wanted a new empty canvas for a certain purpose, but did not necessarily want to go to the grid to find an empty one. Intentional interfaces reduces the cognitive burden of the previous version by removing the necessity of choosing the new canvas's location. Instead, the user creates a new empty canvas by clicking the ``new canvas'' or ``copy canvas'' buttons (depicted in Figure \ref{figure:calico-version-two:canvas}), which each take the user to a new canvas, and in the latter case, populate the new canvas with a copy of the content from the original canvas at which the user was working.

The user's recent navigation history is also recorded to help them navigate canvases. The user may use the backwards and forwards buttons (depicted in Figure \ref{figure:calico-version-two:canvas}) to navigate among the history of visited canvases without returning to the cluster or wall perspectives. When the user returns to the cluster or canvas perspective, the most recently visited canvas is highlighted with a blue halo (as in Figure \ref{figure:calico-version-two:ii-cluster-branch}).

\begin{figure*}[tbh]
  \centering
  \includegraphics[width=16cm,keepaspectratio]{./figures/CalicoVersionTwo/ii-canvas-breadcrumbs}
  \caption{The breadcrumb bar in the upper-left enables users to navigate between canvases without using the wall or cluster perspectives.}
  \label{figure:calico-version-two:ii-canvas-breadcrumbs}
\end{figure*}

The breadcrumb bar at the top of the canvas, depicted in Figure \ref{figure:calico-version-two:ii-canvas-breadcrumbs}, further helps the user in navigating between their canvases. Visible in both the canvas and cluster perspectives (but not the wall perspective), the user can tap on a canvas to reveal its associated canvases. From these canvases, the user can further tap again to directly navigate to any canvas, cluster, or the wall. For example, in Figure \ref{figure:calico-version-two:ii-canvas-breadcrumbs}, the user has tapped ``9. Mise - Figure 2 original'', which revealed a drop-down that displays canvases 9.1 through 9.4 (all of which are also depicted in Figure \ref{figure:calico-version-two:ii-cluster-branch}.

\subsection{Implementation of highlighter}

The fading highlighter enables the user to draw on the canvas without modifying its contents. The fading highlighter creates a dark-golden stroke that is 3px wide, and begins to disappear after 4 seconds until the stroke created with the fading highlighter is deleted. The strokes created with the fading highlighter are visible to everyone connected to the same Calico server and using the same canvas, and the stroke will also disappear and be deleted in the same intervals on remote machines. The fading highlighter supports users in performing mental simulations (design behavior 8), reviews of progress (design behavior 9), and explaining sketches to other designers (design behavior 12), particularly if the other designer is participating in the discussion over a network in a remote location.

\begin{figure*}[tbh]
  \centering
  \includegraphics[width=9cm,keepaspectratio]{./figures/CalicoVersionTwo/highlighter}
  \caption{Highlighter}
  \label{figure:calico-version-two:highlighter}
\end{figure*}

\section{Examples}

In this section, I propose how the features will support the design behaviors in practice by presenting hypothetical examples of their use.

As an example of how these features support design behavior 1, examine Figure 1. Shown is a sketch that includes representations UML class diagrams, user interface mockup, and lists of requirements. Scraps \& connectors provide the building blocks to compose class diagrams and relate them to one another, and stacking scraps on top of one another allows one to represent containment within a class diagram. The user interface mockups are also sketched on top of scraps, where parts of the user interface element can be repositioned because they are scraps.

To show design behavior 2 as performed with scraps and connectors, I remodel the example from Figure X in chapter Y (from background for instance) in Calico.  Figure 2

For the second design behavior, designers will likely use scraps with little content, while occasionally creating very detailed scraps, because they will only add detail when they need it. Figure X presents an example of a set of scraps representing a set of software components, but only one scrap, ``Mobile vehicle'', has additional details inside of it. Further, designers will only add notational detail when they need it, such as the cardinality of ``5'' present next to ``Mobile vehicle''.

Design behavior 3, they refine and evolve their sketches over time, is supported by scraps and connectors because they enable plain sketches to be refined into scraps in a step-wise fashion. The example in Figure 3, for instance, shows how a list of components can be refined into a UML class diagram. In Figure 3b, the list of components first have a box that is hand drawn around each of them, not yet utilizing scraps. The components in Figure 3c are converted into scraps using the press-and-hold gesture. In Figure 3d, the components are arranged, related, and stack on one another to show containment. 
     
Design behavior 4, they use impromptu notations, is supported by scraps and the palette because they allow the designer to save and reuse visual icons to the palette. Figure 4, for instance, includes specialized tags and icons, such as the ``car'' field, which is dropped on top of other scraps in the canvas to signify that those other classes contain the car field. The palette holds common elements and fields, enabling the designer to rapidly reuse those elements without needing to recreate them each time the designer needs them.

For the fifth design behavior, moving between perspectives, designers use a combination of scraps and intentional interfaces in order to move between perspectives. Within a canvas, they use scraps to create representations of different types, where scraps help them to move those representations out of the way in order to create more space for additional representations. Between canvases, intentional interfaces aid designers in moving between perspectives created on different canvases by using the cluster view, the back and forth navigation buttons, and the breadcrumb bar.

For the sixth design behavior, designers use scraps, the palette, and intentional interfaces in order to move between alternatives. Within a canvas, scraps help designers generate and move between alternatives by using scrap copy to generate alternative versions of a scrap quickly. Between canvases, designers use the palette to transfer scraps, or to copy a canvas in order to generate an alternative of an entire canvas. Using tags, designers specifically tag the new canvas as an alternative.

For the seventh design behavior, designers use both scraps and intentional interfaces in order to help them in moving from one level of abstraction. With scraps, they move to a deeper level of abstraction deeper by creating a copy of scrap and then enlarging it make space for more details, or by creating a larger scrap around the existing scrap, as in Figure X. With intentional interfaces, they use multiple canvases to step through different levels of detail in the software architecture of a large system. Given that the total space necessary to sketch the architecture of a system spans more than one canvas, depicting pieces across multiple canvases helps in exploring the parts that the designers find interesting. Those canvases are further explicitly tagged as a level of abstraction.

For the eight design behavior, designers use the fading highlighter in order to mentally simulate over a sketch. As depicted in Figure X, they trace their finger or pen across the sketch without leaving marks, helping them think through the problem.

For the ninth design behavior, scraps and intentional interfaces aid designers in juxtaposing sketches. Using scraps, they rearrange sketches within a canvas so that the sketches are juxtaposed against one another. Further, scraps make sketches larger to better compare them, or smaller to create empty space for juxtaposing other sketches. Using intentional interfaces, designers zoom in to canvases in the cluster view, and further move canvases next to one another in the cluster view to compare and reference them, as depicted in Figure X.

For the tenth design behavior, they designers use intentional interfaces to review their progress. From the cluster view, designers arrangement their canvases so that related canvases are next to one another, which helps them in reviewing work that spans multiple canvases. The tags on canvases further reminds the designers of their intent when creating canvases, for example if the canvases represented alternatives, levels of abstraction, or different perspectives. Also, the history navigation buttons in the canvas, which allow designers to navigate backwards and forwards through the set of visited canvases, help them trace their steps.

For the eleventh design behavior, they use intentional interfaces to retreat to previous ideas. Intentional interfaces supports this design behavior by allowing the design to create as many canvases as they need, and relieve them of needing to erase past work. Intentional interfaces further provides several levels of grouping, such as grouping canvases into clusters, and by linking canvases using tags. These methods of organization help the designer in navigating back to old content.

For the twelfth design behavior, they use intentional interfaces to support them in switching between synchronous and asynchronous work. By working in the same canvas, designers work together synchronously on a design, in which the strokes made by one designer is immediately visible by the other design occupying the same canvas. If the second designer is stricken with inspiration for a new idea, that person copies the contents of the current canvas into another canvas, where they explore the new idea asynchronously.

For the thirteenth design behavior, designers use the fading highlighter to explain their sketches to one another. When two or more designers are occupying the same canvas, the first designer uses the fading highlighter to trace over their sketch, which the other designer immediately sees on their own canvas. With this feature, designers explain their sketches by circling or underlining parts of their sketch for emphasis, or by tracing their finger along box-and-arrow diagrams to demonstrate the flow of data.

For the fourteenth design behavior, designers use intentional interfaces and the palette to bring their work together. With intentional interfaces, they copy the contents of their canvas, and refer to the contents of their previous canvas in order to merge the content. With the palette, they merge their content by copying scraps from several canvases onto a single canvas, and generate a new design on the final canvas..

These examples do not represent the full range of expected behavior, but they provide a high level overview of how the features in Table \ref{table:calico-version-two:designbehaviors} will support their respective behaviors.

\section{Implementation Notes}

Calico Version Two's implementation consists of approximately 100,000 lines of code written in Java 7.0. Calico was developed using the Eclipse development environment, and was tested on Hitachi FX-Duo Starboard interactive whiteboards. It is portable across several operating systems (i.e., Windows XP/Vista, Linux, and Mac OSX), though all of my laboratory evaluations described below were performed using a Starboard connected to a Windows machine that ran Calico. ASUS EEE121 tablets were also used.

\subsection{Architecture}

The architecture of Calico Version Two is based on a client-server architecture. A single server instance manages the content of all canvases, all clients connect to the server transmit inputs from their respective user, and the server broadcasts all updates from clients to all other connected clients. The server is strictly a headless application that listens for clients on a pre-determined port, and establishes a synchronous TCP connection all connected clients. In order to reduce the amount of network bandwidth used during usage, Calico Version Two mostly (but not strictly) uses smart clients, in which much of the computational work in Calico, such as geometry for resizing and rotating scraps,is done on client machines and the results are sent to the server. In our own usage, a single Calico server instance is capable of handling up to 20 users sketching simultaneously before performance is severely impacted. The aforementioned scenario involved usage by students in a classroom scenario.

In order to manipulate the state of the canvas, Calico Version Two uses a command pattern. An individual command consists of a byte array, which is formally called a CalicoPacket within the architecture. The CalicoPacket is Calico's own binary format for transmitting information, and is more compact than other more verbose messaging protocols such as XML and JSON. Both the client and the server contain a matching master list of all possible commands called NetworkCommand class. In order to perform a command, such as a user drawing a pen stroke, a CalicoPacket is created that contains the pen stroke command, the command is interpreted by a singleton command interpretation class, which converts the CalicoPacket from raw byte array data to usable values, and performs the command. The CalicoPacket is used to send and receive information from the server.

The client uses an architecture that is inspired by the MVC design pattern, with some slight variations. The client interface is built on top of the Piccolo2D architecture. In this framework, we build on the construct \textit{PCanvas} to represent the container of all elements, and \textit{PNode} to represent all objects on a PCanvas. In our architecture, both the PCanvas and PNode classes serve the purpose of both the model and the view. There is a further set of controllers that manage manipulating the model, and painting the representation of the model to the view. Each model element may always be represented as a CalicoPacket. 

The core set of classes that form the content in Calico are \textit{CCanvas}, \textit{CStroke}, \textit{CScrap}, and \textit{CConnector}. The container for all sketches is represented by CCanvas, which itself inherits from PCanvas. A CCanvas may contain any sketch element, which includes the set of elements that extend PNodes. These elements that extend PNode are CStroke, CScrap, and CConnector. A Scrap may contain both strokes and other scraps. A connector has a one-to-two relation with scraps, in which the head of a connector is associated to a scrap and the tail another.

In order to manage input from the pen, a visitor design pattern is used to accommodate a flexible amount of modes (drawing, erasing, and highlighting), with a flexible number of unique objects. The interface for all objects that are traversed in the visitor pattern include: \textit{actionPressed}, \textit{actionDragged}, and \textit{actionReleased}. When receiving input from the pen, an input controller first processes the input to determine the appropriate handler that implements the traverser pattern, then forwards the event to that handler. Both the canvas and the scrap implement the visitor design pattern to uniquely handle input. When the visitor design pattern visits the canvas handler, a stroke is created on the canvas. When it visits a scrap, the stroke is drawn directly to the scrap. A benefit of using a visitor pattern for handling input is that future plugins may provide new input handlers to Calico. For example, the scrap-list was first added as a plugin and uses a custom handler for organizing elements.

The architecture further contains a plugin design pattern supported by an event handler and event dispatcher. The motivation behind the plugin architecture of Calico was to partition developer contributions to Calico's codebase. By modularizing Calico's components, several individuals could more easily contribute code to Calico without creating conflicts each others' contributions. Both the palette and intentional interfaces are developed as plugins to the system. In order for a plugin to extend the system, a plugin may register itself as a listener to specific commands. In turn, when a command is performed by the command interpreter, the event handler broadcasts the event to all plugins, which may choose to operate on that action.

%%% Local Variables: ***
%%% mode: latex ***
%%% TeX-master: "thesis.tex" ***
%%% End: ***
