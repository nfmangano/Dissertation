\chapter{Motivation}

\section{Design Behaviors}

Before we introduce the design behaviors that we aim to address in our research, it is useful to return to the definition of a design behavior and explore it in greater detail. A design behavior is a recurrent, recognizable set of actions serving a single purpose within a design meeting. We make four important observations about this definition:

•	recurrent – A design behavior can be observed to happen consistently across many design meetings and across many designers. A particular sketch seen once or twice in a meeting does not qualify as a design behavior. It must be a repeated and general.
•	recognizable – A design behavior stands out as a coherent set of actions within an overall design meeting. The actions clearly belong together and can be distinguished from other groups of actions.
•	set of actions – A design behavior necessarily unfolds over time with multiple actions that a designer undertakes. A single stroke or gesture does not qualify.   
•	serving a single purpose – A design behavior has a purpose in the overall exploration of a design problem and its potential solutions. The set of actions contribute to furthering is exploration.

With this definition in hand, we examined both the software design literature and the broader design literature for design behaviors. What we found was that the general design literature is more mature than the software design literature, in that it has identified quite a few design behaviors that span across different design disciplines. For instance, designers across building architecture, engineering, and product design use constraints to guide their design thinking [8]. As another example, designers in multiple fields use visual similarity (i.e., how they foresee the final product) in their sketches to help them imagine their final product [13].  
The software design literature is only now beginning to catch up to the topic of design behaviors, with the emergence of studies that are beginning to look at software designers “in action” (e.g., [1,5,12,35]. These studies, thus far, confirm the behaviors seen in other design disciplines, but at the same time do not confirm all of them yet, simply because the number of studies remains small. In the below, we include a subset of the design behaviors found in the general design literature, quite a few of which have already been confirmed in the software design literature (i.e., Behaviors 1, 2, 3, 4, 6, 7, 8, 9, 10, and 13) and some of which we simply hypothesize will be confirmed in future (i.e., Behaviors 5, 11, 12, and 14).  We feel confident in making this hypothesis because of our own informal observations of software designers in action, particularly in studying the videos of the SPSD 2010 workshop [1]. While we have not performed full studies of those videos expressly for the purpose of corroborating the general design literature, informally we have seen instances of all of the behaviors we discuss in the below. 

We separate the fourteen behaviors we intend to support in three categories, each of which we detail below.

2.1	Kinds of sketches software designers produce
The first category deals with what the designers draw. The sketches that designers make at the whiteboard are typically not the goal in and of themselves and, as such, the types of sketches that designers make will vary depending on their current design activity. Sometimes, the sketches are used to help the designer in their thinking, by externalizing their ideas and thoughts onto the whiteboard [26]. Other times, the whiteboard is simply a medium to explain a thought, idea, or design in progress. The developer is not using it for problem solving, but instead to communicate information to a listener or collaborator [15]. Because of these different purposes, what designers draw is dependent on what they work on at what point during a design meeting. This leads to the following design behaviors.

1)	They draw different kinds of diagrams. In order to explore a design problem, software designers sketch many different types of diagrams, often within the same canvas [5,17]. They may sketch, for instance, entities and relationships, interface mockups, scenarios, architectures, and other kinds of diagrams [5]. The freedom to sketch multiple kinds of diagrams in the same space is fundamental to supporting the exploration of the design space, as it enables designers to explore an issue from different angles, at different levels of abstraction, or even in different ways altogether. For example, imagining the user interface may help the designer in determining the data model. Thoughts about the data model, in turn, may lead to improvements to the architecture. This behavior also aligns well with the observed phenomenon that tools which restrict designers to use one notation hinder the design exploration and lead to fewer alternatives being considered [16]. 

2)	They produce sketches that draw what they need, and no more. Of the many sketches that software designers create, few are drawn in full detail. Software designers typically can get what they need from a quick and incomplete sketch, e.g., a barebones user interface or boxes-and-arrows sketch [44]. The benefit of a low-detail sketch is that it can be created quickly, and modified easily, giving the designer more rapid feedback [5,35]. Further, providing too much structure too soon can create unconscious barriers to change, resulting in a less exploratory and a less broad search for candidate solutions [45]. This behavior further breaks down into two parts:
a.	They only draw what they need with respect to the design at hand. Low-detail sketches tend to “incorporate relevant information and omit the irrelevant” [43], including only as much detail as necessary to advance the designer’s thinking. When talking, they will only draw what they need to reinforce what they want to communicate [35]. When thinking, they will only draw the details relevant to the immediate issue to help them reason [12].
b.	They use only those notational conventions that suit drawing what they need. Sketches only include as much notational convention as the designer needs in a given situation [35]. For instance, if a sketch can express an idea using only boxes-and-arrows, then no more will be drawn, but if a sketch must represent a hierarchical relation, then a richer array of arrows will be present, typically following the convention of an existing formal notation.

3)	Over time, they refine and evolve their sketches. The level of detail that designers want in their sketches varies over time. Early on, they may need very little (as per the above), but later they may need much more as they expand on their ideas [34]. As designers discuss and work through an idea, they will evolve their sketches with additional decisions and details to support the design process. As part of this refinement process, the sketches will contain more visual precision, and the designer will rework the design to fix any inconsistencies [9]. This behavior, too, breaks down into two parts:
a.	They detail their sketches with increasing notational convention. As a designer’s understanding of the design space matures, so does the representation that they use. As we already discussed, while the designer is aware of the full expressive powers of formal notations, they only borrow from those notations what they need at the time. However, as the designer progresses and the decisions firm up and additional, less-critical decisions are made, they tend to use more and more of the formal notational convention to represent their commitment to the chosen direction [33]. Note that this move towards a more formal notation is not a strictly uniform activity, as different parts of the design may exist at different levels of maturity [35].
b.	They appropriate a sketch in one notational convention into another notational convention. Refinement of sketches does not always mean refinement toward and in a single notational convention. Sometimes, designers appropriate one kind of diagram into another [12]. For example, what may begin as a list may become a boxed list, then boxes with arrows, and finally evolve into a UML class diagram. Separately, at roughly the same time in the exploration, the same boxes and arrows might refine into a user interface diagram. The designer in all likelihood did not foresee this, but in working out their design in place, they re-appropriated the sketch to suit their needs [27]. 

4)	They use impromptu notations. Designers do not exclusively work with the notational convention they know (e.g., UML, ER, etc.), but also, at times, will improvise in the moment. The deviations that they make from standard notations, such as annotating UML diagrams with custom symbols, are deliberate additions that break convention to capture insights before an idea is forgotten. Beyond such annotations and minor deviations, developers also will sometimes adapt wholly new notations on the fly. These often relate to the problem domain that they are explaining, since few domain-specific notations exist, but shorthand is still needed to support the design process [12]. 
2.2	How they use the sketches to navigate through a design problem
The second category pertains to how designers use sketches to navigate the design space. While designers may create many different sketches over the course of a design meeting that vary in detail, notation, and what they represent, there is typically a thread of thought that relates the sketches and the ideas represented in them to one another. 

5)	They move from one perspective to another. Software designers create many sketches through which they shift their focus between perspectives. For example, a designer may create a sketch that explores the user interface of a component, while following that sketch with another that explores its architecture or data model. The designer uses each new perspective to better understand how the parts of a design fit into the whole, asking questions such as: “[what] if we look at it like this, from this angle, it fits together like this” [35]. Each perspective presents a new way of looking at the same design, and what may be subtle in one perspective, may be more pronounced and easier to understand in another. 

6)	They move from one alternative to another. In a sufficiently complex software design task, a software designer will generate sketches of competing solutions before committing to a particular choice [47]. Expressing alternatives as sketches rather than simply mentioning them aloud allows designers to manage their focus and more effectively explore alternative solutions [30]. Once created, designers can compare alternatives and weigh their trade-offs [3]. They may shift their attention back and forth between alternatives and adopt ideas proposed in one into another, or synthesize the ideas of several alternatives into an entirely new alternative [22]. 

7)	They move from one level of abstraction to another. Software designers move between different levels of abstraction, either by “diving into” parts of their design to explore them in more detail or by shifting “back up” to the higher-level representation. This happens often in software design, as many of its notations are hierarchical in nature. A software architect may shift their focus from working out how software components interact with each other to choosing a component and working out how it functions, perhaps by drawing its internal architecture and diving in even further. This behavior typically leads to a multitude of sketches that together consider different abstractions simultaneously [35]. Many scenarios requiring shifts of abstraction have been documented, including the design of user interfaces [36], web pages [14], and so on.

8)	They perform mental simulations. Software designers use mental walkthroughs to gain insight into the consequences of their design [46]. They may need to understand how information flows among components, or inspect their design by mimicking how an end user would interact with it. The software designers ‘interrogate’ their design by testing it against hypothetical inputs and scenarios, often marking over their existing sketch while simulating. Through these mental exercises, the designers can bring to light their implicit assumptions and expose flaws in the design [35]. 

9)	They juxtapose sketches. In order to compare and contrast ideas, software designers will often juxtapose sketches across perspectives, alternatives, and abstractions [35]. A class diagram may be examined in parallel with a sequence diagram to aid the designer in determining how a message is passed between components. The juxtaposed diagrams help the designer in reasoning how the design might work, using the knowledge gained from one diagram to help identify the omissions or mistakes in another, as well as any inconsistencies between them [35]. 

10)	They review their progress. Not all time spent during design is dedicated towards producing new content or verifying whether the design does what the designer intends it to do. At some point, the designers must take stock of what they have done. They momentarily take a step back, away from the design, and consider the progress that they have made and what they have yet to do [27]. They may return to the problem statement or list of requirements and mark off everything they have done to address it, they may generate a new list of issues that they further need to address, or simply just talk amongst themselves, to assess where they are.
 
11)	They retreat to previous ideas. Periodically, designers may reach a stopping point in the exploration of their current set of sketches, such as when they become stuck or simply have exhausted an alternative. They then may choose to return to a previous state of the design (and its sketches) to start anew [46]. For example, an abandoned proposal for a time-based architecture may become a more lucrative option if an event-based architecture proves too costly in system memory usage. In returning to past ideas, the designer may bring new insights and a matured understanding from the exploration they just exhausted, which they can use to explore the past ideas further.

2.3	How they collaborate on them
Software design is a highly collaborative activity, especially at the whiteboard where sketching and design exploration is almost always performed in collaboration with others. The behaviors in this section result from the collaborative aspects of working toward a single vision of the design that is shared by all parties.
12)	They switch between synchronous and asynchronous work. While much of the work that takes place at the whiteboard is typically synchronous, with all participants focusing on a single aspect of the design they are discussing, it is known that designers occasionally break away to explore an idea independently while the others continue with the main discussion [11]. This typically occurs when a sudden inspiration strikes, or when a designer wishes to develop a counterexample or alternative to what is being discussed now.

13)	They explain their sketches to each other. After any independent work takes place, and even in cases where one designer drawing on behalf of the group and “has the floor”, the designers must synchronize their mental models of the state of the design [12]. This behavior relates to Behavior 9, but represents its collaborative version. They will need to verbalize their mental simulations to explain the consequences of a particular choice, clarify the meaning of a sketch, or even simply explain their assumptions or inspiration. While explaining, they may sketch on top of the existing diagram, using their marks to guide attention or add detail, or simply gesture over the diagram if they do not want to edit the sketches.

14)	They bring their work together.  Sometimes, as a result of asynchronous work, the designers need to integrate their ideas from separate sketches into a unified design. This may involve bringing parts of a sketch over, creating a new sketch that integrates both, or sometimes working on a third alternative that combines the best aspects of each but requires a different underlying approach to make that work [12].

\section{Research Question}

Now that we have introduced this set of design behaviors, we return to what it would mean to build support for it. We first recognize that there is a spectrum of ways in which we could support the design behaviors. On one end of the spectrum lies the whiteboard itself, a minimally intrusive, informal medium that supports only drawing and erasing of content. The naturally occurring behaviors are permitted, but not necessarily supported, which is the basic problem that this research is trying to address. On the other end of the spectrum is the formal design tool, a highly structured, heavy-weight environment with hosts of explicit features. Examples of such tools are Rational Rose [37] and ArgoUML [38]. Theoretically, they support a number of the behaviors, such as, for instance, Behavior 5 by providing multiple views on the same model, or Behavior 6 by maintaining many projects. They, however, do not support all, and the ones that they do are not nearly as fluidly supported as necessary. 

Between the informal whiteboard and the formal tool lies a range of possible approaches that provide different blends of the strengths of each. This leads to our first research question:
What minimally invasive, coherent set of features can be designed that is sufficient to effectively support these behaviors? 

It is useful to examine the phrasing of this research question:
•	minimally invasive – We seek to build support that does not completely abandon the whiteboard experience. People go to the whiteboard for its fluidity and flexibility. Any solution that we design must preserve the feel of the traditional whiteboard as much as possible.
•	coherent set of features – We wish to arrive at a set of features that build on each other using a unified set of design principles and metaphors. Building an isolated feature for every behavior would not satisfy our research goal. 
•	sufficient to support all of these behaviors – Each behavior should in some way be supported by the overall set of features, and support for any particular behavior should not come at the cost of another. 

Having put forward our research challenge, we also need to evaluate what the effect of having targeted tool support is on the design process and the designs produced. This is a remarkably deep question, particularly since the literature to date has documented that these design behaviors take place, but has not yet fully articulated their effects on the design process and design product. For instance, switching among perspectives (Behavior 5) seems to have a positive effect on the eventual design [2], and, in certain cases, it has been documented that the consideration of multiple alternatives (Behavior 6) also seems to lead to a better design [3]. However, even these studies are hard pressed to provide absolute answers. The first study does not examine the optimal length of time a perspective should be explored: is thirty seconds too short, thirty minutes too long, or what is the general distribution? Similarly, the second study does not talk about the number of alternatives: is three sufficient, should twenty be explored, or does it depend on the design problem in some way? These are questions that we cannot objectively answer at this time. 

For this reason, our evaluation must be exploratory. An obvious evaluation would examine factors such as the frequency of design behaviors, time spent working in each behavior, and how our tool impacts the interleaving of the design behaviors. A more comprehensive evaluation would attempt to correlate the appearance of these design behaviors with the quality of the process and the quality of the design by, for instance, relying on outside experts to rate the designs that are produced and correlating this quality with the frequency, interleaving, etc. of the design behaviors. While this would possibly lead to a conclusive statement with respect to the impact of our tool, it would be very challenging to perform this evaluation given that a typical design exhibits complex interleavings of many design behaviors, making achieving any statistical relevance difficult. 

Our evaluation will take a middle ground. Using the basic evaluation of frequency, time spent, and interleaving as the basis of our comparison, we will identify a subset of behaviors with noticeable differences from the traditional whiteboard, and analyze the impact of those differences; that is, we want to focus on what appears to be most interesting differences, and examine in more detail why these differences emerged and how they influenced the designers’ work. While the differences will likely pertain to the existing set of fourteen behaviors, we of course do not rule out that new behaviors may emerge, as enabled by the features of the software, that might be worthy of study. 


%%% Local Variables: ***
%%% mode: latex ***
%%% TeX-master: "thesis.tex" ***
%%% End: ***
