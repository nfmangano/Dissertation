\chapter{Discussion}
\label{chapter:discussion}

In the previous chapter, I reported on the experiences of three different groups using Calico. In this chapter, I bring these observations together to answer the research questions posed in Chapter \ref{chapter:research-question}.

The rest of the chapter is organized as follows. In Section \ref{discussion:strengths-and-weaknesses}, I review the strengths and weaknesses of each feature, drawing upon how they were used across all three field evaluations. Section \ref{discussion:cog-dim}, takes a step back from the observations and examines the theoretical potential of Calico's features by performing a cognitive dimensions analysis, revealing insight into why features did and not work. Section \ref{discussion:overall-strengths-weaknesses} then examines the overall strengths and weaknesses of Calico, examining the benefits and drawbacks of using Calico in a work environment. Section \ref{discussion:interviews} discusses the feedback from the users, their personal experiences, and what they saw as important in their experiences. Section \ref{discussion:summary} then summarizes the contributions of this chapter.

\section{Strengths and weaknesses of each feature}
\label{discussion:strengths-and-weaknesses}

In this section, I examine each feature on its own merits to discuss their strengths and weaknesses. Table \ref{chapter:discussion:strengths-weaknesses} provides an initial summary of this section, and each feature is examined in detail in the subsequent subsections.

\begin{center}
\begin{longtable}{|p{3cm}|p{6cm}|p{6cm}|}
\caption{The set of design behaviors and the features that supported them}\\
\hline
\textbf{Feature} & \textbf{Strengths} & \textbf{Weaknesses}\\
\hline
\endfirsthead
\multicolumn{3}{c}%
{\tablename\ \thetable\ -- \textit{Continued from previous page}} \\
\hline
\textbf{Feature} & \textbf{Strengths} & \textbf{Weaknesses}\\
\hline
\endhead
\hline \multicolumn{3}{r}{\textit{Continued on next page}} \\
\endfoot
\hline
\endlastfoot
\hline
Scraps \& 

connectors &
%strengths
1. Flexible nature is useful in creating many representations relevant to software

2. Gestures enable quick manipulation of hand drawn sketches and scraps without mode switching

3. Existing hand-drawn boxes could be refined into scraps

4. Bubble menu is self discoverable
 &
%weaknesses

1. Requires training to use properly

2. The press-and-hold gesture for select and moving scraps was slow for continuous arranging of a large number of scraps

3. Could not change scrap color or border, text scraps were ``visually heavy''

   \\
\hline
Palette &
%good 

1. Bootstrapped design sessions by importing existing artifacts into multiple canvases

2. Used to build set of reusable icons in storyboards

3. Used as a global clipboard to copy content across canvases in order to juxtapose them

&
%bad 

1. Hard to find scraps in palette when it has many items (more than twenty)

2. Users could not add plain sketches to the palette, must be turned into a scrap first

\\
\hline
Intentional 

interfaces &

1. Clusters provided a simple metaphor to separate content of different projects

2. Linking canvases allows a narrative to be constructed from a set of canvases

3. Made free space to sketch immediately available because they can immediately jump to new canvas or copy a previous one

&
%bad
1. Moving between the canvas view and the cluster view was not a smooth action, canvases were too small to visually distinguish after moving

2. Users had a difficult time understanding where a new canvas appeared in the cluster view

3. Difficult to juxtapose content across canvases

 \\
\hline
Fading 

highlighter &
%good
1. Supported explaining sketches and mental simulations

2. Unlike traditional pointers, fading stroke allows for transient annotations such as arrows, underling, circling, etc.

&
%bad 
1. Strokes made with highlighter are anonymous

\label{chapter:discussion:strengths-weaknesses}
\end{longtable}
\end{center}

\subsection{Scraps}
Scraps, on the whole, were a relatively successful feature in Calico, and a significant improvement of their Calico Version One counterpart. Scraps replaced the lasso functionality to manipulate sketches, which enabled users to manipulate content without changing modes. Scraps also depicted software representations such as software components, process flows, and user interfaces. They sometimes were used to represent lists, though not often. While there are opportunities to provide better support, such as the ability to change its fill and border color, scraps saw a significant amount of use across all groups. 

A strong quality of scraps was their flexibility in representing diagrams. While scraps were not always used in favor of regular whiteboard sketching, they were used often and yielded benefits such as helping diagrams evolve more gracefully and arranging content. Actions which become natural when using scraps, such as annotating it with a color patch, underline, are not straightforward in other mediums. The same actions could be done with drawing tools, but arranging sketches would be more complex. Alternatively, one could design using a more formal tool which would allow for easy arranging of sketches, but the tool may not be able to freely add annotations or symbols. The flexibility afforded by scraps led to the unexpected behaviors in the experiment, such as when the interaction design group tagged them using color and the research group color coded their scraps as well. In comparison to Calico Version One, which saw a relatively narrow set of representations created with scraps in experiments, Calico Version Two yielded a much greater variety of representations created with scraps. It is possible that the longer term evaluation led to more time to become accustomed to using scraps; however, it is more likely that the revised functionality of scraps led to this improvement.

What scraps lacked, however, was more expressive power in their visual features. Users could not change the blue color of scraps and their outline, which many users requested the ability to do. Users in the studies commented that they wished to create text scraps without the blue boarder, as they felt that it made the sketch ``visually busy''. In another session, a user wanted to create text scraps with text of different colors in order to encode meaning, but instead used hand-written text in order to use colors. While these and other features may have been useful, their absence did not obstruct the sessions as users found ways around them, and users remarked that these sessions were provisional.

Scraps gestures, such as creating, selecting, and moving, were a strong and weak point for scraps, depending on the situation. Particularly with long-term users, scrap gestures became a powerful tool to quickly move and copy content. When working with box-and-arrow diagrams for software, they allowed content to be rearranged quickly. When working with lists, they allowed individuals to adjust spacing in handwritten text or make more space. In user interfaces, they enabled rearranging of elements to simulate using the sketched interfaces. Unfortunately when used to categorize large number of scraps, as the interaction designers did, the press-and-hold gesture to select and move scraps were perceived as too slow. The interaction designers, in this case, requested a separate mode specifically targeted at moving scraps.

A weakness of scrap gestures was that they required some practice to become fluid. Gestures such as the landing zone to create scraps were enabled quick manipulation, but new users required some training before they could trigger the gesture reliably. 

Scrap gestures in Calico Version Two, however, was a large improvement on the scrap gestures in Calico Version One, which were not discoverable and were reported as unpredictable. In contrast, the bubble menu in Calico Version Two was reported as being straightforward and discoverable. Further, the landing zone used to create scraps was self discovered by users, while the equivalent action in Calico Version One was also difficult to discover. 

Lastly, the ability to convert regular hand-drawn sketches into scraps, i.e., refine them, further benefited teams. This addressed a significant issue in Calico Version One, which was that individuals often did not use scraps because they would first hand draw a box, and later draw a scrap around the box, resulting with a hand-drawn box inside a scrap. While not used often, users at field sites did use it for both refining existing sketches into scraps and as a recovery mechanism when they failed to trigger the ``landing zone'' gesture correctly.  Users further refined their scraps by using list scraps, which organized scraps into a linear compact list. 

\subsection{Palette}

While the palette significantly improved on the palette of Calico Version One, it still remained one of the weaker features in Calico Version Two. This was due most in part because the palette did not noticeably support impromtu notations for software oriented groups. However, while not performing strongly in this specific design behavior, the palette succeeded in other aspects, such as supporting impromtu notations for storyboards and acting as a global clipboard to transfer content between canvases.

The palette performed strongly in supporting users in three scenarios. First, the palette served to bootstrap a design session for interaction designers by importing several dozen images from outside of Calico. The imported images drove the design session which spanned several days. Second, the palette actually did serve to store a set of graphical icons that were improvised by the interaction design group in their design sessions. Members from the interaction design group found these icons useful in creating storyboards. Third, all groups used the palette to juxtapose new sketches against previously created ones. They did so by using the palette as a global clipboard to copy old sketches to the canvas with the newly created sketch.

Some aspects of the palette could have used further improvement. Users had a difficult time locating items in the palette when it contain in excess of twenty scraps. The interaction designers imported dozens of images of faces, which were difficult to distinguish because of their small size in the palette. Also, users wished to add sketches to the palette without first making the sketch into a scrap. Users reported that making the sketch into a scrap made it too heavy, e.g., they simply wanted the sketch without a scrap's blue background and border.

A potential weakness of the palette was that users designing software systems did not use it to perform the fourth design behavior, inventing impromptu notations. However, this was not a weakness of the palette, but instead caused by two factors: 1) users found it faster to redraw certain sketches because it was faster, and 2) other Calico features were redundant with the palette. With respect to the first point, users often only needed to sketch the name of a component, which could be created much more quickly manually using scraps than finding the component in the palette. With respect to the second, the availability of the copy functionality in numerous forms lessened the need for the palette. For example, when users sketching a software system wanted to create an alternative or move perspectives, they often used the ``copy canvas'' button in intentional interfaces, or used the ``copy scrap'' button in a scrap's bubble menu.

\subsection{Intentional interfaces}

Intentional interfaces was a relatively successful feature in Calico. It advanced on the concept of the grid from Calico Version One, and satisfied numerous design behaviors in doing so. It provided qualities that were not available in the grid interface, such as names for canvases, relationships, and order of work performed. Intentional interfaces, however, had some weak points as well. Navigation was sometimes clumsy in the cluster view, juxtaposing sketches across canvases was difficult, and understanding the relationships between canvases was sometimes difficult for new users.

Many of the grouping mechanisms in intentional interfaces improved upon the grid interface of Calico Version One. While the grid was well received by users of Calico Version One, users increasingly desired ways to separate and categorize their content. Intentional interfaces addressed this problem in several steps. First, clusters provided a generic method to divide canvases into topics, which helped partition content between different projects in Calico.

Second, linking canvases provided another level of organization of canvases, while at the same time offering an ordering to the canvases. Providing an ordering to the canvases was shown to be beneficial in practice as it allowed a set of canvases to compose a narrative. A sequence of canvases could show a walkthrough of a software architecture as well as the exploration of the design space across multiple perspectives, abstractions, and alternatives. Further, grouping canvases by linking them helps users better recall design sessions and understand the content of their sketches several months afterwards.

A third benefit of intentional interfaces was the immediateness of free space. In comparison to the regular whiteboard, users felt more at ease in creating more sketches because they could always create more space and return to previous sketches. The presence of both the ``new canvas'' and ``copy canvas'' buttons made moving to a new sketch a ready-at-hand action, which they used to explore different perspectives and new alternatives.

Users encountered a few issues using intentional interfaces as well. First, the movement between the canvas and the cluster view was not a smooth transition for the user. When loading the cluster view, the perspective always zoomed out, which made the individual canvases too small to distinguish. Second, users had a difficult time understanding the shape of the cluster when creating new canvases from within a canvas itself. One member of the OSS group described the shape of the cluster as ``wizardry'', in which it was a mystery to him as to where new canvases would appear. Third, users could not easily compare contents between canvases.

\subsection{Fading highlighter}

	
The highlighter feature worked well for a particular set of scenarios. The fading highlighter stands on its own as users did not combine its use with other features, but it was useful in scenarios that the other features did not support. Particularly, it supported users in mentally simulating over their work, and also in explaining concepts to others. 

The second strength of the fading highlighter was that it enabled transient annotations in verbal walkthroughs and explanations. Traditionally, verbal explanations of sketches include either a simple pointer, like a baton or laser pointer, or the speaker must permanently mark up a sketch. The fading highlighter provides a compromise of the two approaches, allowing the speaker to use symbols that convey more meaning, such as drawing arrows between sketches, underling them, and circling key points for emphasis.

While the fading highlighter addressed these situations well, it could still be improved to provide better support the two design behaviors that it targets. First, it could provide better support for explaining sketches by providing better awareness features. For example, the strokes could somehow show who is making the stroke. Second, mental simulations could be better supported if the sequence of strokes were somehow recorded. Many of the mental simulations explain how data is moved between components. If this explanation could be captured, and later played back, particularly with audio explanations, the design rational within the design sessions could be played back.

\subsection{Summary}

Each feature brought forward a set of unique advantages not available in the other features. Scraps were strong in supporting the kinds of sketches designers drew. The palette acted as a global clipboard and made sketches reusable across canvases. Intentional interfaces provided many ways to organize, partition, and make free space for sketches. The fading highlighter supported verbal explanations of sketches well. While each feature had usability issues, such as navigating in intentional interfaces and scraps requiring training to use properly, the features provided a net benefit to their users.

\section{Cognitive dimensions analysis}
\label{discussion:cog-dim}

Having examined the use of Calico's features \textit{in practice}, this section now examines the \textit{theoretical} support of those features by performing a cognitive dimensions analysis. The cognitive dimensions framework exposes the affordances that a notation or medium offers. Analyzing Calico's features through this framework provides a basis to determine how those features help or hinder the notation that is implied within Calico. In order to perform the cognitive dimensions analysis, I look at each dimension, and discuss factors that either increase, decrease, or, all things being equal, provide equivalent support for designers as compared to the whiteboard. These findings are summarized in Table \ref{table:discussion:cognitivedimensions}.

%Cognitive dimensions is the golden standard of what Calico should be. Is it possible.. not if they did it. CD exposes to what uses a notation or tool leads.

%Table \ref{table:discussion:cognitivedimensions} summarizes the findings in this section. In most cases, Calico's features provided benefits to one or more cognitive dimensions. Intentional interfaces played a large role in finding relevent canvases in designs diffused across canvases. Scraps and the palette made sketches less viscious and easier to manipulate. Overall, Calico improved on the regular whiteboard for the following dimensions: abstraction, hard mental operations, premature commitment, provisionality, secondary notation, viscosity, and visibility. Calico also had some shortcomings for some of the groups. These can be attributed to inefficient support for the cognitive dimensions of consistency and error-proneness. Support for all other dimensions, including closeness of mapping, diffuseness, hidden dependencies, progressive evaluation, and role expressiveness received moderate support.

\begin{center}
\begin{longtable}{|p{3cm}|p{4cm}|p{4cm}|p{4cm}|}
\caption{What CDs Analysis Highlights about Calico}\\
\hline
\textbf{Cognitive Dimension} & \textbf{Factors that cause an increase over the whiteboard}& \textbf{Factors that are equal to the whiteboard}& \textbf{Factors that cause a decrease to the whiteboard}\\
\hline
\endfirsthead
\multicolumn{4}{c}%
{\tablename\ \thetable\ -- \textit{Continued from previous page}} \\
\hline
\textbf{Cognitive Dimension} & \textbf{Factors that cause an increase over the whiteboard}& \textbf{Factors that are equal to the whiteboard}& \textbf{Factors that cause a decrease to the whiteboard}\\
\hline
\endhead
\hline \multicolumn{4}{r}{\textit{Continued on next page}} \\
\endfoot
\hline
\endlastfoot
Abstraction	
& %increase
Scraps help visualize abstractions and make them manipulatable
& %equal
You can draw anything
& %decrease

\\
\hline
Closeness of 

mapping	
& %increase
Can manipulate notations at the level of scraps and connectors
& %equal
Can draw in any notation
& %decrease
Annotations, like cardinality, do not move along with connectors
\\
\hline
Consistency	
& %increase
Can save and refer back to previous sketches, on whiteboard they may have been erased
& %equal
Must be socially enforced
& %decrease
Difficult to socially enforce when content is diffuse and requires hard mental operations, as opposed to whiteboard where content is all in one space
\\
\hline
Diffuseness	
& %increase
Calico breaks up the whiteboard space across several canvases
& %equal
Small designs stay on one board
& %decrease

\\
\hline
Error-proneness	
& %increase

& %equal
No error checking
& %decrease
Manipulating and automated copying of existing sketches may prevent errors caused by redrawing sketches
\\
\hline
Hard mental operations	
& %increase
Design may be diffuse in Calico and details of sketches scattered across many canvases
& %equal
Small designs are contained within one area
& %decrease
Content can be imported into Calico using image scraps; content can be positioned side-by-side using scraps; palettes allow users to copy content for reference across canvases; navigation button allows user to return to previous content quickly
\\
\hline
Hidden dependencies	
& %increase
Content may be diffuse, making hidden dependencies hard to find
& %equal
Small designs have all potential references in one area
& %decrease
Intentional interfaces helps manage diffuseness by grouping content together to make hidden dependencies easier to find
\\
\hline
Premature commitment	
& %increase

& %equal
Medium is considered provisional; rejected decisions can be preserved by crossing them out rather than erasingor deleting them
& %decrease
Easier to explore alternatives by copying existing ones using scraps, palette, and intentional interfaces
\\
\hline
Progressive evaluation	
& %increase
All sketches are preserved and can be returned to; fading highlighter allows evaluting without marking the original sketch
& %equal
Users must manually judge the ``rightness'' of a sketch by manually reviewing it
& %decrease
Diffuse designs are harder to progressively evaluate; fading highlighter marks during an evaluation cannot be later referred to as regular marks could be on a whiteboard
\\
\hline
Provisionality	
& %increase
Users can make copies and explore alternatives without losing original ideas; more sketches can be reviewed across canvases in intentional interfaces; fading highlighter allows sketching over design without permanently marking it; canvases can be tagged as alternative using intentional interfaces
& %equal
Medium is considered provisional by users; anything can be changed at any time
& %decrease
More difficult to evalute sketches diffused across canvases; marks from progressive evaluation no longer remain and cannot be referenced
\\
\hline
Role expressiveness	
& %increase
User has unlimited space to fully express all parts of a design
& %equal
User is limited by the space available
& %decrease
Designs spread across canvases have higher diffusion, lower visibility, and may require hard mental operations to reference other canvases
\\
\hline
Secondary notation	
& %increase
Second notations can be attached to scraps and move with scraps; custom shape of scrap can be used; color of connectors can be used to capture meaning
& %equal
Users can sketch in any secondary notation
& %decrease
Annotations outside of a scrap do move with a scrap and the annotation is left behind
\\
\hline
Viscosity	
& %increase

& %equal
Users can sketch anything they want; a full whiteboard is less likely to be changed because it contains many important decisions
& %decrease
Scraps make sketched content not ``locked in'' such that can easily be moved and manipulated; gestures make manipulating scraps a fluid action
\\
\hline
Visibility	
& %increase

& %equal
Sufficiently small designs are entirely visible on the whiteboard
& %decrease
Designs diffused across multiple canvases have very low visibility, leading to hard mental operations
\label{table:discussion:cognitivedimensions}
\end{longtable}
\end{center}

\subsection{Abstraction}
Abstraction refers to Calico's ability to support representations at different levels of abstraction. Since Calico at its most basic supports plain sketching, it inherently allows maximum flexibility in what can be represented, much the same as the traditional whiteboard. 

Calico, however, improves on the regular whiteboard because users can manipulate those abstractions at the level of scraps or, when using intentional interfaces, entire canvases. On the whiteboard, the designer can only draw and erase strokes. In Calico, they have scraps to depict abstractions, which support quick movement of the drawn abstractions, implicit grouping by stacking scraps, and have arrows that remain attached to the scrap as it is moved. Intentional interfaces enables the user to explore different levels of abstraction across multiple canvases, and supports the user in navigating between these levels of abstraction by linking canvases. 

When the content of a design session is suffuciently small, the whiteboard may have a better expereince than Calico because all abstractions are available in one space, while in Calico they may be diffused across several canvases. When on a single whiteboard, the designer can immediately look at all levels of abstraction at once. If the content of the design grows beyond what can be sketched on the whiteboard, then Calico becomes more useful, as it can capture all sketched content across several canvases, whereas on the whiteboard that content would need to be erased, or scattered across multiple whiteboards, posters, or pages. 

\subsection{Closeness of Mapping}
%You can draw in any notation. Factors that increasing is that you can manipulate notations at the level of scraps and connectors. factors that decrease it is annotations like cardinality... they don't move along with connectors.

%Very similar to the whiteboard. Compared tot he whiteboard, all that is preserved. Now I've given people the way to manipulate them as wel.

Closeness of mapping refers to how closely representations map to the referred concepts. Calico is similar to the whiteboard in this regard in that designers are at liberty to sketch any representation. If the designers decide that there is a more suitable representation, they are able to create a new sketch that depicts that representation simply by drawing its corresponding shapes. These shapes, then, can be turned into scraps when so desired. 

When using scraps, Calico has the positive consequence that representations can be manipulated as objects, which may more closely map to the mental model of the designer than plain strokes would. Further, where on the whiteboard representations may need to be erased and redrawn to create a sketch that closely matches the desired notation, scraps are reused by copying them. 

A negative consequence of using scraps is that certain visual visual annotations, do not move along with scraaps and connectors. This may cause sketches to become less closely mapped to their concept as cardinality annotations may become disassociated from their target sketch. This may cause additional work as users will need to manually move annotations with scraps, or simply leave the annotation behind after the scrap is moved. 

%
%- people could get very close to what they are representing. Importing screenshots into the environment is a big one.
%- interaction designs could manipulate their objects directly, no indirect names
%- software people could import snippets of code
%- researchers imported screenshots of user interfaces, source code, etc.

\subsection{Consistency}
Consistency refers to the degree to which features of structures and syntax are used the same way throughout. Calico does not provide an automated method to enforce consistency. As with sketching, maintaining consistency between elements and sketches is up to the users to ``socially'' enforce \cite{Petre2013BookChapter}. If all content can be condensed to a single whiteboard, the designer can refer back to previous sketches to check for consistency. On Calico, since the designer is encouraged to use multiple canvases, it may be more difficult to manually verify consistency across all canvases.

However, if the design is too large for a single whiteboard, Calico would provide some help over the whiteboard. Where on the whiteboard sketches would need to be erased, the designer may save them in Calico, and refer back to them to verify consistency between old sketches and new.

%- consistency brought by calico comes from copying, none beyond other, unless enforced by personal discipline.
%- palette supported this, but seldom used this way
%- multiple perspectives enforced consistency,
%- researchers used multiple documents, hard to maintain consistency across those

\subsection{Diffuseness}

%There's a cost with Calico - it makes things more diffuse. By taking the original whiteboard, cutting it into pieces, I've actually made it a little bit harder (speaking of Calico). 
%
%Counter argument is that people work with it different. They partition it differently. The jury is out. 
%
%Can't say there's moderate support for diffusement, instead say there's moderate support for reducing diffuseness.
%
%I don't have as much as much space in an individual Calico canvas as the whiteboard. If everything can be sketched on the whiteboard, it's superior because I can see everything at once. However, if the design grows beyond the space of a whiteboard, diffuseness becomes obliviousness on the whiteboard because what is not immediately visible has been erased. Calico incrementally becomes more important because diffuseness on the whiteboard has become obliviousness, sketches are no longer exist, but in Calico they do. However, in Calico there is now the navigation problem among these things. The structure of intentional interfaces is one way we address that. It is unclear... it doesn't erase diffuseness, it doesn't moderately support it, but what it does do is acknowledge it, and we have built features to address that. More research is required to determine how well the features combat diffuseness, the support is there.

Diffuseness refers to how much the meaning of sketches is spread out across multiple sketches. Calico potentially increases diffuseness because the potential for an unlimited number of canvases may encourage designers to break up their designs across multiple canvases. On the whiteboard, designers are forced to limit their design sketches to the limits of the whiteboard.  If everything can be sketched on the whiteboard, then it is superior because designers can see everything at once. 

However, if the design grows beyond the space of a whiteboard, Calico's functionality helps in dealing with diffuseness. On the whiteboard, diffuseness becomes obliviousness because what is not immediately visible has been erased. Calico incrementally becomes more important because, while on the whiteboard sketches no longer exist because they have been erased, in Calico they remain. Yet, this raises a new issue, which is the problem of how to navigate between sketches in Calico. The structure of intentional interfaces is one approach to managing the diffuseness. It is unclear if intentional interfaces effectively addresses diffusenss, as it does not remove it. What it does do is acknowledge it, and gives the user tools to work around manage it themselves. More research is required to determine how well the features combat the consequences of increased diffuseness.

%Further, enabling one to copy canvases and transfer content using the palette encourages diagrams to be diffuse. However, intentional interfaces mitigates these issues by helping to navigate designs diffused across different canvases. By providing links between those  canvases, users can more easily find content immediately relevant to the current canvas. Intentional interfaces does not provide a fully comprehensive solution to diffused diagrams, but it does moderately support it in a lightweight fashion.

\subsection{Error-proneness}

Error-proneness refers to the degree to which a notation induces ``careless mistakes'' \cite{Petre2013BookChapter}. As with plain sketching, Calico does not provide any safe guards against errors. Individuals are free to improvise and switch between notations, which may increase their tendency to perform errors if there is no clearly defined standard notation. 

While Calico does not alert the user of when they have performed an error, Calico's features do automate certain actions which may reduce the likelihood of errors. Scraps and connectors, for example, provide flexibility in creating representations by enabling the moving, resizing, and rotating existing sketches. They maintain the structure of objects as they are moved, which may help when working with box-and-arrow diagrams. In comparison to plain sketching, scraps and connectors help prevent errors by making additions to representations, like process flow diagrams, less tedious so they do not require diagrams to be redrawn when repositioning elements, as one would have to if sketching at the whiteboard. On the other hand, designers recognize that there is value in redrawing a sketch from scratch because each time a sketch is redrawn, its contents are re-evaluated \cite{petre2009insights}, and copying existing sketches removes this opportunity.

%The research group benefited from the use of scraps, which helped avoid errors by removing much of the tedium in working with their state diagram. When working with their state diagram on the whiteboard, they reported that it had become too large to manage. After moving to Calico, they had more flexiblity in managing their space. They were able to create more space by moving text-scraps, which moved all connectors with the scrap as well and retain the shape of the diagram. Further, the flexiblity to create connectors with custom paths, as opposed to straight lines, made connectors more legible. 

%Overally, scraps and connectors helped prevent errors by making adding additions to the state diagram less tedious, and not requiring diagrams to be redrawn to reposition elements.

%No safeguards against errors, other than what was socially enforced by the designers.
%
%- repeated elements helped avoid using the wrong elements, and reviewing content in meetings helped catch errors


\subsection{Hard Mental Operations}
%By making it electronic, and breaking it into pieces, you gain, but you lose something too. On the whiteboard, when people start to erase, you gain.

Hard mental operations refers to information that needs to be referenced or is nested. Similar to the cognitive dimension of diffuseness, in a simple or small design, both the plain whiteboard and Calico are equal in that all parts of the design are visible and there may be little need to reference outside information. However, in larger projects, Calico may have more hard mental operations because the design may be diffused across several canvases requiring the user to reference sketches outside the immediate space. Users may need to use the intentional interface feature to move between canvases to reference other sketches, or use the palette to move content so that it can be referenced in the active canvas. For a large design on a plain whiteboard, content may need to be erased, and Calico may provide better support because that content would be preserved.

Calico further reduces hard mental operations in other sutiations. It allows users to both import content and copy content to the canvas they are currently in. First, image scraps allow users to import existing artifacts into a canvas to refer to while sketching. For example, a developer may import source they are working on, or a diagram created in another tool. Second, scraps allow users to move content so that it is positioned side-by-side to what they are currently working on. Third, the palette allows users to copy content between canvases so that they can reference old content in the new space. Fourth, the navigation button in intentional interfaces allows users to return back to the most recently visited canvas. While this may still quality as needed to refer to parts of the design outside of the immediate visibility, the feature reduces the effort needed to do so.

%In the case of the OSS group, the developers needed to reference information that was spread across several canvases. Members from their team reported that they did not need to do so often all members were already familiar with the depicted system. When they did need information to be available more immediately, they copied content onto the target canvas. The developer working with source code did so by pasting screenshots of his code into his canvas. In the case of the interaction designers, they pasted pictures of people they interviewed, however they needed to reference their notes from their interviews. The researchers, however, had a greater need to perform hard mental operations because of the scale of their state diagrams. The researchers who originally developed the system eventually internalized the diagram to a degree that they no longer needed to reference it, however the researcher that needed to be onboarded created additional diagrams, such as tables, in order to help him mentally step through the state diagram. In his case, he mitigated the need to reference other canvases by copying the pieces of the state diagram that he needed into his own canvases.

%Overall, the ability to copy content within Calico somewhat reduced the need to reference other canvases. Scraps, intentional interfaces, and the palette were helpful for both importing screenshots and copying content to the immediate working area.

%How often did designers need to reference elsewhere and nest?
%
%- the OSS group had to incur hard mental operations by referencing across diagrams because of multiple levels of abstraction
%	- they moved to high levels of abstractions, and use cases
%	- hard to interpret what was there because had to keep track of everything
%- the interaction design group had to do lots of hard mental operations because interviews not represented. Also placing along axies, and mult. perspectives
%- researchers experienced a lot of hard mental operations because of the terseness of the diagrams (paths, etc.)

\subsection{Hidden Dependencies}

%On the whiteboard, everything is there. I can glance over on the whiteboard, but I can't in Calico.

Hidden dependencies refers to representations that are dependent on one another, but the dependencies are not visible because they are either not declared or not in a person's field-of-view. As observed in past studies \cite{dekel2007notation}, sketches that cross several spaces result in having dependencies on other spaces that are not explicitly stated. Sketches that begin in one canvas in Calico may extend onto another canvas, or across mediums, such as their own computer or paper, if the users are using Calico as a complementary tool. This could occur, for example, when canvases depict different perspectives or abstractions. 

The intentional interfaces feature in Calico helps in declaring and finding these hidden dependencies on other sketches by providing features that group canvases of related content. Clusters in intentional interfaces help group canvases into sets of similar topics. Linking canvases into chains provides a second level of grouping that help designers in finding related sketches. These features do not solve the issue of revealing all hidden dependencies, but they do improve working with them by by making those dependencies easier to find.

\subsection{Premature Commitment}
A notation or representation that causes a user to \textit{prematurely commit} means that it requires a person to make a decision before they have the information they need \cite{Petre2013BookChapter}. Sketches and content produced in Calico are typically used to support conversations or thinking through a design, and thus there is less of a concern to prematurely commit because all content is provisional. The sketchy appearance of content within Calico visually re-enforces the notion that content is provisional. Also, the ease of changing content, such as drawing an ``X'' symbol over rejected content, allows for decisions to be rejected without permanently discarding the idea. In this regard, Calico is equivalent to the whiteboard which also has these qualities of provisionality. 

However, Calico further reduces premature commitment in comparison to the whiteboard by making sketches more easily copied and reproduced. Features such as scraps, the palette, and intentional interfaces make it easier to explore alternatives. For example, while on a whiteboard, a user may be hesitant to erase a large figure or deviate because they do not want to lose previous work. In Calico, they can simply create a copy of it in a new canvas, or continue their work in a new blank canvas.

%Members from all groups reported not feeling pressured to commit to decisions within Calico. The OSS group reported that they valued the ideas and tasks generated from using Calico rather than the sketches themselves. They reported that they sometimes immediately revisited sketches after meetings in order experiment with alternatives before returning to their desk to implement a new change. The OSS group did email snapshots of sketches from Calico to themselves, but reported that these emails were archival and served to remind them of the ideas generated. The ideas were not considered committed until they were implemented into the system. In the research group, the members used the sketches from Calico in support of implementing their software system and writing an academic conference publication. One of the researchers remarked that nothing in Calico was permanent until ``it was written in the [academic conference] paper''. 

%Overall, the provisionality of content and the ease with which it can be changed, i.e., low viscosity, lowered the tendency for content to be prematurely committed in Calico. However, there is very little gain in this regard over the whiteboard. The ease of copying content reduces the tedium in exploring alternatives, but this was only marginally beneficial for the groups with respect to preventing premature commitment.

%Calico provided a sketching look and feel to elements. Users consciously commited to decision by converting handwritten text into rectangular scraps and sometimes formal connectors.

%- there is not a sense of commitment to what is on the board.
%- research group turned to calico/whiteboard because it gave the sense of freedom from commitment

\subsection{Progressive Evaluation}
Progressive evaluation refers to the ability to run a simulation or obtain feedback on a representation that is only partially complete in order to determine its correctness. While Calico does not directly support progressive evaluations, it does support users in manually judging the ``rightness'' of a sketch by supporting the reviewing and explanation of sketches. Much like the whiteboard, designs may be manually reviewed and discussed by close inspection. 

With large designs, Calico provides some additional benefit with intentional interfaces, which aids users in navigating between canvases to review them. However, as the sketch grows too large, it becomes progressively more difficult to manually evaluate all sketches. Further, the collaborative aspect of Calico allows remote designers to obtain feedback, during which the fading highlighter would help during group reviews to discuss content. In local designs, the fading highlighter further helps progressive evaluations by making the sketches cleaner, evaluative marks are not left behind using the fading highlighter. However, this could be considered a loss in comparison to the whiteboard as well, as designers could potentially refer back to these marks to remind them of their evaluation.

%The intentional interface feature supported the OSS group in reviewing their contents by allowing them to zoom out and switch between canvases quickly. The OSS group reported that they would switch to the cluster-view, zoom in to the canvases they had been working on, and pan while discussing them. During group meetings, they further used the fading highlighter during discussions and evaluations of components, drawing paths taken by data between components during their explanations. The researchers, similarly, engaged in verbal dialog to review their designs. The researchers engaged in review of their designs, particularly the state diagram that summarized their software system. They stepped through the state diagram to verify its correctness, and also used it in support of evaluating the completeness of their software system. They used the sketch to keep track of what was and what was not implemented, and updated their state diagram with this information.

%Overall, Calico supported individuals in performing a manual evaluation of their designs. In this regard it could be considered similar to the whiteboard, but intentional interfaces provided the benefit of reviewing multiple canvases, and also the quality that Calico was an online shared resource that was shared by everyone. 

%- all groups used intentional interfaces to review, summarized, and proceed. different perspectives

\subsection{Provisionality}
Provisionality refers to the quality that indecision or options can be expressed \cite{Petre2013BookChapter}. Given that sketching is the primary mode of operation in Calico, individuals are free to break away from formal notation to declare multiple alternatives. Users can always erase a sketch to change a value, or simply cross out a sketch so that a design can be rejected without visually removing it from the design space. Calico goes further in this supporting provisionality by supporting the quick generation of alternatives using duplication features such as scrap copying, canvas copying, and the palette. With intentional interfaces, uses can explicitly express entire canvases as being an alternative by tagging it as such within intentional interfaces.

%The OSS group used the fading highlighter to propose ideas about sketches that they were not sure of yet. After discussing a box-and-arrow diagram using the fading highlighter for more than twenty minutes, they changed elements within that diagram. They did so by drawing a large ``X'' over the component, and wrote the name of an alternative component next to the component with an ``X'' on it. In another instance, members in the OSS group used intentional interfaces to create a copy of a canvas, and tagged the canvas an ``alternative''. They did so on multiple occassions, such as during a group discussion, and also in individual design sessions. The research group expressed options in another fashion. In their case, they performed a major refactoring of their system, effectively diverging from the design in their previous sketches. Rather than deleting these old sketches, they moved to another cluster to sketch the design of their refactored system. They considered the contents of the old cluster as an archive of the first version of their system. 
 
%Overall, Calico provides some improvements in provisionality over the whiteboard. Within a canvas, Calico is much the same as a whiteboard in that the user has the flexibility to sketch alternative names. Across multiple canvases, intentional interfaces helps manage options at a larger scale by providing tags to label canvases as alternatives. Further, the ability to copy and create new canvases and clusters encouraged more sketching, and possibly exploration of alternatives.

%Everything in Calico was considered provisional and outside the formal specfications. Provisionality was reduced by copying content onto another medium. Teams walked away from designs, and return to them in order to be reminded.

\subsection{Role Expressiveness}

Role expressiveness refers to a reader's ability to see how the parts fit into the whole design, and how those parts relate to one another \cite{Petre2013BookChapter}. In other words, a reader can independently interpret what each part in a sketch does without further information. Given that users have the flexibility to sketch any notation they need, the limiting factor in both the regular whiteboard and Calico is the space available to create representations. With enough space, the role of each part can be better understood if all details are visible. On the whiteboard, this means that a sketch must be sufficiently small. In Calico, this means that there is a virtually unlimited amount of space, but the user must then content with a high amount of diffusion and low visibility. Intentional interfaces provides the means to manage the space needed to explore a design across multiple sketches to depict all parts of a design.

With Calico, there is the potential to address this issue using intentional interfaces by using several canvases to include all parts of a design. However, there is a further issue. Within the context of the informal design activities that Calico supports, designers typically only draw as much as they need to, which results in partial diagrams with minimal detail. As such, designers may create representations that have poor role expressiveness due to minimal detail. Very often, the original designer may need to be consulted in order to understand their sketch. However, content in sketches may be easier understood by looking at related content and hidden dependencies, which the intentional interfaces feature may help the designer do by navigating between canvases linked into chains. 


\subsection{Secondary Notation}

Secondary notation refers to the use of formalisms that deviate from primary established notations. Because of the freeform nature of sketching within Calico, users have the flexibility to overlay additional detail in sketches and use improvised notations as needed. Within Calico, users may choose to represent concepts using scraps with connectors as an alternative to basic sketching, in which case they can use the shape of scraps, as well as the colors of connectors to overlay meaning. Users may add annotations to scraps, such as drawing a symbol in the corner of a scrap, text content in the scrap with a color that has an assigned significance, or by using connectors in creative ways.


\subsection{Viscosity}

Viscosity refers to a medium's resistance to change. Similar to a regular whiteboard, the contents in Calico exhibit a low viscosity because content can always be erased and redrawn in any manner. 

Calico has three features which potentially reduce viscosity in comparison to plain sketching: content is moveable, gestures make manipulating content faster, and content can be copied. First, all sketched content can be moved using scraps, as opposed to the whiteboard where it must be redrawn. The capacity to move content means that sketched content is not ``locked in'' like a whiteboard, but instead content is less viscous because it can be adjusted. Second, with sufficient familiarity with scrap gestures in Calico, moving content can potentially be done quickly with minimal effort. A user may be less likely to modify content if accomplish that task requires a great deal of effort. In Calico, users do not need to switch modes or move their hand away from sketches, allowing them to move content fluidly without distracting them away from plain sketching. 


\subsection{Visibility}

Visibility refers to how easily all parts of a design are visible, or at least how easy it is to juxtapose two parts of a design \cite{Petre2013BookChapter}. As with a whiteboard, a design is highly visible so long as it is small enough to fit within a single space. In Calico, designs that are diffused across multiple canvases will have much less visibility than designs that are limited to a single canvas. The intentional interfaces feature attempts to address this issue by making it easier to navigate between canvses, and also to zoom out to the cluster view and view all canvases at once, albeit at a low-detailed perspective.

%Also, juxtaposing parts of a design is easier in Calico due to scraps, the palette, and intentional interfaces. Scraps allow users to reconfigure a diagram such that two parts can be placed next to one another by moving them or copying them. The palette allows parts of the design to be copied to other canvases to compare them. Intentional interfaces increases the ease of moving between parts of the design, and organizes the canvases so that it is easier to step through a design. Further, two canvases can be placed next to one another in the cluster view for comparison.

\subsection{Summary}

Overall, the major benefits of Calico over the whiteboard are that intentional interfaces allows more content to be preserved whereas on the whiteboard it would be erased, and the fluidity of manipulation offered by scraps. The benefits make content less viscous, more provisional, promote secondary notations, role expressiveness, and reduce error proneness. What is lost in these features is that designs may be more diffuse, and because of that diffuseness, require more hard mental operations, have less visibility, more hidden dependencies, and be harder to progressively evaluate. For each benefit gained through features, Calico loses the simplicity of the whiteboard, but attempts to mitigate these losses through the other features.

\section{Overall strengths and weaknesses}
\label{discussion:overall-strengths-weaknesses}

\section{Interviews with users}
\label{discussion:interviews}

%In the previous sections, I found that Calico's features adequately support the broad set of design behaviors and discussed the strengths and weaknesses of those behaviors. In this section, I corroborate those findings with a deeper discussion of the features. I do so by first performing a cognitive dimensions analysis of Calico's features at a hypothetical level. I then review the feedback from the interviews.



\subsection{Interviews}
%//I've already made my interviews, now it's just my opinions
%It's clear that each group is using Calico very differently. They're finding different strangths and weaknesses. Interactions designers... not having an infinite canvas. When push came to shove, they made a lot of use of Calico. 
%There is the open source group, they didn't make big use of scraps, but they did represent code. Not our intention, but big use. One of the members used code because he had a large whiteboard.
%The research group. One of the guys did a lot of prep work.
%//the important part is that there is a lot of different uses of Calico, and Calico held up. All designs were created. 
%//Design group stated that not having an infinite canvas was a hurdle
%//what we got from the researchers were comments like this:
%//what we got fromt he oss group were like this:
%//if you pull out real quotes, it's stronger.
%//roll context into discussion. 
%
%Characterizing each session

In the previous section, I examine the capability of Calico's features to support design from the perspective of a cognitivie dimensions analysis. In this section, I discuss the feedback received on the features from the participants interviewed in each group.

From the interviews and observations, what is clear is that each group used Calico very differently. Each group used Calico to support different activities, with different amounts of people, and in different settings. These groups subsequently found different strengths and weaknesses in Calico.

The interaction designers used Calico to simulate activities that would have normally involved physical artifacts over a physical whiteboard. As such, many of the comments that came from the interaction design group related to Calico's strengths and weaknesses as they compared to the physical artifacts that the interaction designers normally used. For example, a strength that the interaction designers found was its ability to import images and arrange them as they would have on a physical whiteboard. Explaining why this was important, one member of the interaction design group stated, ``It's really hard to say 'that person or this person'. Much easier to speak to a face''. They found that Calico's strengths were those that enabled actions not easily done with physical artifacts. For example, replication of artifacts encouraged the interactions designers to explore more perspectives than they would have had they used actual physical artifacts. Calico's features further allowed them not only arrange the images themselves, but also the sketches generated around those images, in which they could arrange and copy written text to organize the whiteboard. 

However, they found some experiences in Calico still not on par with using the physical whiteboard. They viewed the gestures of moving scraps too slow, where one member of the interaction group stated that ``the overhead in manipulating was too much''. Further, they found the space available in canvases being too small compared to the boards they normally use, stating that they were ``blocked by the physical limitations of the [electronic] board''. Despite these obstacles, they overcame such obstacles, making dozens of canvases with sketches and scraps. They made real progress in a real design task within their professional work.

The OSS group took on a very different approach to Calico, using it in group design discussions of three or more people, and also personal sessions over both software components and software code. A strength of Calico in this setting was to increase their capability to do more in the same space. For example, intentional interfaces encouraged members to brainstorm more, where one member of the group states, ``we're more willing to draw any random thing on there because we know that we can erase it, or go to a new canvas at any point. I would say more random ideas get thrown on there.'' The OSS group in particular took advantage of free space to create dozens of canvases. For example, they consecutively chained canvases within intentional interfaces to generate sets of canvases that together formed narratives, such as how data steps through software components, or how a user interface behaves. Some of their use was also unexpected. The sessions involving source code came as a surprise, as I did not expect the developers to drop paste screenshots of actual code into Calico, but these sessions became some of the most interesting, as they involved a great deal of impromptu notations to step through source code.

%``I had something where I needed to get my thoughts written out as far as I was trying to design...''
%``It was really useful to switch... I used a lot of color. It kind of helped me identify, like, which kind of objects are important. The other thing was that I wanted to look at the XML code that was relevant to this diagram, so I went on my machine, logged in over the web browser, took a screenshot of my code, and pasted it in to the canvas. That way when it came in over here on the board, I could look at the code from the screenshot, and do my work based on that. So that was very helpful.''
%``Having it in one page, one screen, was helpful.''
%``I ended up taking a different direction in the strategy I was working. So I just copied the canvas, I made an alternative, yeah, so that was nice. I made another screenshot of different code.''

The research group presented a third unique setting in which meetings were distributed, had a member that did a lot of design up front before the group design sessions began, and generated very large process diagrams that described the entire behavior of their software system. As opposed to the other groups, the research group members were more isolated from one another, and used meetings to review their work, plan what to do next. The research group viewed one of Calico's strengths in their setting as being able to carry on meetings with remote members, where one member stated: ``I think the biggest benefit was crossing space and time. Being able to pause and resume'', and ``the main benefit was that after I left, we were able to reference all those things that we created while I was there pretty easily..''. They viewed Calico as more beneficial than photos of a whiteboard, stating that ``you can't go back and edit that.'' Further, one member of the research group prepared his work ahead of time for meetings, stating: ``I could have done this in a powerpoint as well''... ``it's much better [to have done it in Calico] because later when I show it, I can have people changing it, and in powerpoint you cannot sketch and draw.'' This particular member further found the visual structure of intentional interfaces helpful when reviewing work from past meetings, stating ``if you're designing a complex thing with stages and you're trying to tell a story, you can say: okay we've tried that, would you like to see all this path we went through?'' Calico served as a virtual meeting space for the members that helped maintain continuity of meetings across the duration of the project.

%--- christian
%``we had one whole design that was thrown away''
%
%``designs get very complex... you want to keep a history of what you've done, the branches that you've pruned. Having a structure is essential. If you're designing a complex thing with stages and you're trying to tell a story, you can say: okay we've tried that, would you like to see all this path we went through? If you don't have the structure you'll have to create it somewhere else. If it's already here...''

The takeaway was that each session introduced a unique way of using Calico. Each group had a different set of needs, a different setting and setup of people, and yet, Calico held up to a significant portion of their needs. Each group followed through with their design sessions and completed work in actual projects without abandoning their use of the tool. That is not to say that it was a perfect match for all groups. One of the teams, the interaction design group, did say that they would not continue to use Calico because its features did not match their needs, such as an infinite canvas and the system having difficulty with the number of images they used. This may either mean that their needs would be better suited by another tool, or further research is needed within the Calico environment.

Further, from the interviews, three observations stood out: 1) Calico did not interfere with their normal design process, 2) Calico deployments still have issues that need to be addressed that are intrusive, and 3) that Calico led to positive changes in their design habits.

First, the OSS group and the research group reported that Calico and its features did not prevent them from carrying out their design as they normally would have, while the interaction design group reported that Calico somewhat got in the way of their normal design activity. The OSS group reported that they did not feel any loss of expressive control in using Calico in comparison to the whiteboard, and reported that they normally would have performed many of the same activities on the whiteboards available in the immediate area. The research group reported that they formerly carried out their designs on the whiteboard, and transitioned to using Calico for the same activities. The interaction design group reported that it closely approximated the whiteboard, but they found that the quality of sketch input forced them to write large and that the system lagged at times, which was distracting during their sessions.

%Did it prevent you from doing what you normally would have done?
%- quality of sketching on whiteboards was largest compromise. Tablets helped, but was not the same (1,2,3)

Second, there were some issues that were commonly intrusive among the three groups. Nearly all groups reported that the large electronic whiteboards diminished the quality of their own handwriting, forcing them to either write slower, or write larger. The OSS group connected to their Calico server using either tablets or their own computers, which they reported was preferred in producing clearer handwriting due to precise input. All groups used text-scraps to produce legible text as well. All groups also reported that launching Calico on their own machines was sometimes a hurdle. The OSS group reported that it was less of an obstacle for them because of the availability of tablets preloaded with Calico. The research group reported trouble setting up Calico on the laptops of new members of their group. Lastly, the members of the interaction design group were accustomed to a different set of gestures and interaction design patterns from their own set of digital sketching tools, particularly OneNote, which led to confusion with Calico's features, such as how to use scraps and gestures.

%Features that detracted
%- going against the grain... people are accustomed to their existing tools.
%- too slow to bring up
%- turned to word documents to supplement activity

Third, all groups reported some positive changes to their existing design habits. All groups reported that they had a sensation of having more free space, and reported that they created more sketches than they normally would have as a result. The research group reported that they created more complex sketches as a result of scraps and connectors, which helped them address a deeper level of complexity in their design. Both the OSS group and the research group reported that, since they did not feel a need to delete unused sketches, they returned to old sketches more often.

\section{How well the features support the design behaviors}
\label{discussion:designbehaviors}


%simply about design behaviors, and how well calico's features supported it beyond regular whiteboard use.

Bringing together the reports of the design behaviors in Chapter \ref{chapter:evaluation}, Calico did indeed provide support for all fourteen design behaviors. As indicated by the Table \ref{chapter:evaluation:designbehaviors-table} in Section \ref{chapter:evaluation:design-behaviors}, not all features supported their intended design behavior, and indeed some features provided support for behaviors when they were not expected to, but, by and large, every single design behavior exhibited itself, and every single one was done using either scraps, the palette, international interfaces, or the fading highlighter. Not continuously by every single group, but the groups used the features when they could have simply sketched as they normally would have on the whiteboard. 

\subsection{Supporting the kinds of things they drew}

In most cases, scraps \& connectors, as well as the palette, supported the kinds of representations that the OSS group and the research group drew. Individuals from all groups used these features to create an variety of different representations (design behavior 1). The representations they created ranged from very low-detailed boxes with just names to complex representations made from scraps stacked on top of each other (design behavior 2) . While not happening often, some sketches did evolve into from simple hand written lists to complex structures involving box scraps and connectors (design behaviors 3). In many of these complex sketches, individuals created impromtu notations to concisely depict semantically rich concepts (design behavior 4). Together, both scraps and the palette were used along with regular sketching such that they provided the sensation of using a whiteboard, but with significantly lowered viscosity for changing them. In the rest of this section, I discuss the impact that supporting these design behaviors had on the design sessions, as well as cases in which support for the design behaviors did not work as well as they should. 

Pertaining to the first design behavior, scraps allowed several types of representations to grow ``organically''. Representations such as lists, software architecture representations, and user interface mockups received the most benefit from Calico's features. With respect to lists, scraps made handwritten representations more fluid to edit because content could be quickly moved. Temporary scraps made moving content a ``ready-at-hand'' action, which allowed users to make space to insert handwritten items in lists, move content out of the way, and copy repeated elements in lists without pausing their activities to switch modes. With respect to software architecture representations, scraps enabled users to more easily to grow diagrams in place. Users created complex box-and-arrow diagrams, and often moved scraps around the canvas to make space for more parts of the design. The scraps themselves were often very low detail, typically only containing the name of software component, but the relationships between the scraps were heavily discussed and edited. With respect to the user interface mockups, the scrap copy feature helped users more easily explore alternative user interfaces. Users could quickly create variations of a user interface. In comparison to a regular whiteboard, representations in Calico could be more easily iterated upon in-place in Calico, and in comparison to a formal tool, representations have much more flexibility in what they can represent.

In the case of the interaction designers, scraps fell short in supporting the the interaction designers working with the planar representations they created. These representations, however, were outside of Calico's original design scope. In planar sketches, a scrap's position on the horizontal and vertical axes define its relationship to other scraps. For example, the interaction designers drew a one-dimensional axis, and placed scraps according to their ranking on the axis from left-to-right. This is in contrast to box-and-arrow diagrams, in which the relationship between scraps is explicitly declared using arrows. There are numerous ways that scraps could be augmented to better support planar type representations, such as automated support for positioning scraps in one- and two-dimensional axes, tables, etc., but ultimately, the interaction designers were able to create these planar representations without Calico's features getting in their way. 

Pertaining to the second and fourth behavior, scraps \& connectors worked well to support users in conducting rich discussions over the runtime behavior of software components. Many of the conversations that took place were focused on how components pass data to one another, and these discussions were expressed using hand-drawn arrows, Calico connectors, and the fading highlighter. These discussions led to individuals improvising unique symbols in their representations to depict complex concepts. Further, I found it interesting that, while discussions were centered on the flow of data between software components, the users never recorded the paths taken by data onto the sketch itself, nor made any attempt to record it otherwise in their sketches. Formally, these paths could be captured using UML sequence diagrams, yet not once did I see this type of representation created. Based on my observations, I do not believe this was due to lack of foresight by the developers, but instead that sequence diagrams are too specific for the purpose of design at the whiteboard, and do not yield enough benefit to merit the time used to sketch them.

%The research group demonstrated that Calico connectors could be used to successfully create dense process flow diagrams. Gestures made connectors fluid to create, and connectors that moved along with scraps made creating empty space on the canvas a natural action. 

Pertaining to the third design behavior, one of the scenarios in which Calico was the most useful was when individuals approached the board with a vague set of ideas, and used Calico to build new relationships and higher level concepts from that data over time. Much in the way that agile teams use Post-It notes to organize stories or qualitative researchers use affinity diagram to discover topics from data, members from groups used Calico to build relationships and categories from their work. At least once in each group, members turned to Calico with an unstructured set of data, and a goal of what they wanted to build from that data. After iterating over several canvases, they reached their goal by grouping and creating relationships between their data. 
%While not always used, the support by scraps and the palette for the first four design behaviors improved their capacity to build designs over their raw data.
% One of Calico's strengths was to help teams process ``uncooked'' information.

The OSS group, for instance, found this activity valuable in their use of Calico. One member of the OSS group stated:

\begin{myindentpar}{1cm}
\emph{"We have different categories of stuff, and trying to figure out what relates to what. So we kind of just wrote it all out there, and just started almost randomly just drawing arrows to see what was what, we eventually deleted a bunch of connections, moved stuff around to make it cleaner looking."}
\end{myindentpar}

The interaction design group encountered much the same situation within their experience. They did not use scraps to the degree that the OSS group did, but they used Calico's features in other ways to build relationships among their data. They imported images of dozens of individuals that they interviewed, and used this unstructured set of images to explore the patterns that appeared in their interviews. Where the OSS group used connectors to provide structure, the interaction design group instead visually organized faces into clusters according to their similarities in different categories. 

Pertaining to the fourth design behavior, while scraps supported many groups in improvising notations, users sometimes improvised box-and-arrow notations without the use of scraps. In particular, users sometimes chose not to use Calico connectors to show relationships between scraps. The OSS group, for example, did not use formal Calico connectors on many occasions, but rather sketched arrows by hand. While at first glance it may appear that the OSS members did not find Calico connectors sufficient to their needs, it was instead the case that, in those particular instances, the OSS group did not further need to evolve their software sketch. Thus they did not stand to benefit from making their hand-drawn arrows into Calico connectors, which enabled support such as remaining ``stickied'' to scraps as they are moved. In the instance in which the OSS group did evolve sketches involving components with arrows between them, they did indeed use formal Calico connectors. 

%[+]It was a medium to process undigested information. The SW people did it in a meeting. Interaction designers did it. The onboarding researcher did it as well
%- "it ended out pretty different from where it started."
%- "it helped in visualizing things"
%- it didn't provide good support for lists, but at the same time people don't need much support for lists. scrap interaction provided good support for editing in place
%- Support for grouping content for the interaction designers... it did not support some of what they did, but they began to use some of the advanced functionality, such as linking content with formal connectors

%[-] It didn't quite support impromtu notations for software design, but it did for interaction designers.
%- palette did not quite support impromptu notations as I had hoped. Supported interaction designers, but software designers note as much. Software designers used the most creativity in establishing different kinds of connectors. A better feature would have been to include profiles, and define the meaning of profiles.
%- scraps lacked ability to be color coded, which a developed wanted to use to assign meaning
%- sw people use visual variables to assign meaning to their diagrams.

\subsection{Supporting how they navigated their sketches}

While the interaction designers did not adopt Calico's features into their design designers pertaining to navigating their sketches, Calico's features did support the groups who did use the features, and those groups continued to use Calico's features across several design sessions. 

When working within canvases, scraps provided a supporting role in working with perspectives, alternatives, and abstractions. Scraps enabled users to examine different perspectives by moving and manipulating sketches, generate alternatives by creating multiple copies, and also were copied across canvases that represented different levels of abstraction of one another. The fading highlighter also excelled in supporting mental simulations over designs and explaining designers, two design behaviors which were quite often done together.

When working across multiple canvases, intentional interfaces played a strong role in helping people move between perspectives, alternatives, abstractions, reviewing their progress, and retreating to past designs. Intentional interfaces made it easier for users to jump to a new space to continue designing, and provided a structure to the design space as they moved between the parts of their design. The palette did not quite support these particular design behaviors as it was intended, but rather the palette played a role in helping users juxtaposing content across canvases, where intentional interfaces itself did not sufficiently support users. Intentional interface's inability to support juxtaposition also likely led to it only weakly supporting the design behavior involving bringing work together, but intentional interfaces did adequately support switching between synchronous and asynchronous work. 

%With the exception of the interaction designers, the intentional interfaces feature sufficiently supported users in navigating between their sketches. 
%For those that used it, intentional interfaces and scraps both provided support for the fifth, sixth, and seventh design behaviors, which involved moving between perspectives, alternatives, and levels of abstraction.
%- intentional interfaces worked, scraps did too for a different reason
%- palette not so much
%- Again, they did use it to navigate between canvases. The OSS group, half of the research group did use it and found it use full. Other half of research group, and interaction designers did not.
%- <<these were all trying to address the same problem>>
%- "this was nice, being able to copy this scrap, it allowed us to show different use cases, right."

A notable contribution of intentional interfaces was that it provided an ordering to the canvases by linking and tagging them, which had a positive impact on the design sessions. This quality helped members from multiple groups to manage their different perspectives, distinguish alternatives, and step through multiple levels of abstraction. Users did not always remember where exact canvases were located, but they had a sense of where they were located within chains. Establishing an order between their canvases enabled them to frame their designs in terms of a story, in which the canvas closest to the center of the ring presented the initial exploration, and the canvases near the end of the chains contained content that contained final design decisions or summarized content from the previous canvases. Users mentally thought of moving between cells in the chain as moving ``backwards'' and ``forwards'', which, while it helped them form a story, resulted in confusion with the buttons to navigate through the history of visited canvases.
%- So much so, that it changed people's expectations of features, such as the backwards and forwards buttons.

Where the ordering of intentional interfaces fell short was when the canvases did not portray a linear story. For example, a single canvas may have multiple related canvases in which one was an alternative and another was different level of abstraction. Linear canvases did not capture the relationship between these canvases as well they could have. An alternative representation to the linear list may be to represent the list as a hierarchical tree, but this may not adequately capture the full relationship among the canvases as the canvases relate to one another in multiple and cross cutting ways. Any method used to add structure to the canvases should be lightweight to allow these different categories to emerge.

It was interesting to note that intentional interfaces encountered issues with scaling when chaining was not used. One cluster belonging to the research group grew to 34 canvases, but did not use clusters. Subsequently, all members of the research group experienced difficulty finding canvases within this cluster because they could not remember where they were, and the canvases in the radial layout were too small to recognize. In contrast, members of the research group had other clusters with just as many canvases, but could more easily locate content because content within the cluster was grouped into chains. The chains were visually distinguishable at a glance, and members could recall the design sessions in which they created the canvases within particular chains.

While the features met their criteria for support in two of the groups, they fell short in supporting the interaction designers in supporting several of the design behaviors pertaining to navigating between sketches. This may have been due to several situational factors. First, the sketches produced in this particular activity of interaction design may not lead to several of the design behaviors the features were intended to support. Design behaviors such as generating alternatives and navigating between different levels of abstraction are behaviors that have been observed during software system design. The interaction designers were engaged in a different type of design, which was to build a set of personas. This type of design involved generating several perspectives that are independent of one another, whereas a change in one perspective in a software system directly effect other representations of the same perspective. Second, the design sessions of the interaction designers only involved two individuals who shared a single electronic whiteboard. Features such as the fading highlighter, and design behaviors, such as those pertaining to collaboration, may only occur when several individuals participant in a design session, and members are using several devices. Third, the interaction designers already have a strong culture around how they use electronic sketching tools, which has a different mechanism for navigating between content. They use OneNote, which allows for scrolling between documents. This may have conflicted with Calico's method of navigation, as well as their comment that they wanted all of their sketches visible at once. However, despite these problems, the interaction designers were able to create nearly two dozen canvases with different perspectives.

%A few factors contributed to the features not supporting navgiation as well as they could have. Interaction designers were designing in a different domain, leading to different needs. Juxtaposing sketches was difficult, which led to other ixxues. Some features require specific conditions to be necessary. sometimes you need to plan ahead
%Interaction designers didn't use them because...
%- not as complex domain as software
%- also different culture
%- Wanting all parts visible...


A further factor was that intentional interfaces became helpful in longer term projects. The interaction designers did not engage Calico for long term projects, having only used it intensely over the course of a week, and later for a very short period. In contrast, the OSS group used it for several weeks, and the research group for several months. One of the OSS group members commented on the importance of preserving his work for later use, stating that, ``I pretty much always, when I was done, went back to my computer and exported it as a PNG, and saved it so I had a record". The research group further retreated to their past designs after several months had passed, citing that they wanted to refer to decisions that they rejected in the past in order to not repeat the same mistakes in future iterations. 

It is interesting to note that content in Calico was used for a short period of time, but remained on the server for long periods of time. This is in contrast to the whiteboard, which has been shown to have short-lived, ephemeral content \cite{cherubini2007let}. Further, research that examined settings in which whiteboard content could be automatically saved and reloaded later led to individuals enforcing higher quality control on their own content in anticipation of using it later \cite{Branham2010fromwhiteboard}. I did not observe this to be true of sketches in Calico, instead these sketches remained the same quality and members of the OSS group and research group did not change their behavior in anticipation of reusing content. It is possible that this difference came about because users of whiteboard archiving systems require the user still need to erase their whiteboard and can only use images. In Calico, they can go back and make additions, which mentioned as useful by all groups (including the interaction design group, even though they did not return to content).

%Require specific settings.
%[+] Retreating to ideas happens in very long term projects.
%- highlighter is useful with teams and multiple devices



%Bringing ideas together in Calico is not an easy action.
%- very hard to juxtapose items across canvases
%- juxtaposing an important part of bringing together, not as well done
%[-] Scraps and the palette helped juxtapose, but intentional interfaces did not address this design behavior as intented. This also likely affected bringing work together.

\subsection{Supporting collaboration}

On the whole, the intentional interfaces feature served to support the design behaviors pertaining to supporting collaboration. Intentional interfaces supported individuals in moving between synchronous and asynchronous work. The fading highlighter supported extensive discussions and explanations of designs in group settings. For the final design behavior, intentional interfaces did not support bringing together work as well as it should have, however, scraps surprising provided a supportive role in this behavior. Also, the interaction designers did not use any of Calico's features in support of collaboration. This factor was influenced by a work culture of designing in pairs of two individuals, which is too few people to cause collaborative design behaviors, and a strong existing culture for collaborating on content on the whiteboard.

In the cases where intentional interfaces was used, it served to coordinate the workspace, allowing individuals to coordinate working together synchronously, or deviating to a separate canvas to work asynchronously. As stated in the results from Calico Version One in Chapter \ref{chapter:calico-version-one}, users that cannot access the board while someone else is speaking will ``spin their wheels'' and try to hold on to a thought until write their own idea. Intentional interfaces directly addressed this concern, allowing individuals to directly write on their own tablet or jump to their own space by pressing ``new canvas''. The OSS group, who shared a physical space at their office, had the most to gain from this functionality because they could access their sketched in meetings that they generated while on their own. One member of the group stated:

\begin{myindentpar}{1cm}
\emph{``the fact that someone can work with their own tablet or computer, like [a member of the team] did with his alternative view, is something really powerful to do. If you try to do that in real life... we're blessed with two whiteboards, but some companies may only have one. It makes it harder to do that. Especially when someone is already at the whiteboard discussing something, and you want to bring in an alternative perspective but you need to wait until they're done doing whatever they're doing. That definitely makes groupwork easier.''}
\end{myindentpar}

This design behavior directly led to individuals further needing to explain the designs they generated on their own, or spur individuals in meetings to ask others about content they discovered by looking at new canvases they found in the cluster-view. While both the OSS and research group usually gestured with their hands, the fading highlighter was used a fair amount to walk through diagrams. However, it was most often used to explain content that was prepared beforehand, in which the users switched to using the fading highlighter, and did not switch to drawing mode during their explanations. 

Further, the quality of drawing content was important in the use of the fading highlighter in explanations. Users drew arrows at components, circled names of software entities, and wrote words using the fading highlighter. Users from both the OSS group and the research group valued the ability to temporarily draw, as opposed to using a bright-red dot pointer that does not leave a trail, which is the method used by commercial presentation tools such as Microsoft Powerpoint \cite{WinPowerPoint}.

%Supporting synch and async
%- [+] It supports long term collaboration. People used it at both their desks and at the board.
%"being able to work simultaneously is nice."

While the other behaviors were well supported, bringing work together was likely the least supported design behavior by Calico's features was the fourteenth, bringing work together. The expectation was that users would copy and merge content from previous canvases. Instead, users merged ideas verbally and used those conversations to sketch new content. For example, OSS group did so by navigating to previous cells, then drew several use cases on new canvases to consolidate their previous discussions. The research group surprisingly used scraps to bring together work that was done outside of Calico as well. One research group member described using this during their week-long intensive design session: 

\begin{myindentpar}{1cm}
\emph{``...the rest of the week we'd basically be coding along trying to implement all this, sort of diffing what's in the code against this [gesturing to calico]. So we'd glancing back at this trying to understand... sort of like `oh, here's the new pieces that we added and here's the pieces that we will be adding, and refering this, here's the bit that we haven't yet implemented'''}
\end{myindentpar}

It is possible that the intentional interfaces feature did not adequately support bringing together work because of its lack of fluidity in juxtaposing sketches. Users in the OSS group and research group turned to the palette for juxtaposing elements sketches across canvases, but such an action takes several steps to copy the contents to be juxtaposed to another canvas. Alternatively, users also used the back and forward button to navigate to recently visited canvases in order to compare the current canvas with previous canvases, but such a solution puts the burden on the user to remember previous canvases. Instead, the users should have been able to place canvases side-by-side in the cluster view, which would have allowed them to bring work together from multiple canvases simultaneously.








%Did it encourage you to do anything that you wouldn't have done?
%- return to sketches...
%- performed sketches as they normally would have, but was able to make more complex versions (3)
%- all reported that they created more content than they would have on the whiteboard (1,2,3)


%Overall, with exceptions because of hardware limitations and performance issues, all groups reported that the features did not intrude on their normal design activity, and that it brought positive benefits. 
%
%//do something with this later
%Calico played a complementary role to existing tools and practices for each team, serving as an external thinking space for individual and collaborative design. In each case of each team, the members turned to Calico to work on ``wicked problems'' in which the solution was not readily apparent, and required some exploration of the design space before arriving at the solution. Sometimes these problems involve trying to make sense of raw data before a software system is built, as the interaction designers did. Sometimes it is to make modifications to an existing system, as OSS group developers did. Other times they may be building a new system from scratch, as with the research group. Given that the studies each provided glimpses into different pieces of the lifecycle for each group's project, it is difficult to draw a direct comparison of the differences between the approaches of each group. However, several commonolities arose in Calico's role throughout its use for each team. Calico use was not restricted to any single phase of design, but was flexible enough to support groups at different points in their design process.

%\section{Minimally Invasive}
%\label{chapter:discussion:minimally-invasive}
%
%In order to a determine whether Calico's features were minimally intrusive, or at least the degree to which they were considered so, I both report on responses from the interviews, and perform a cognitive dimensional analysis of Calico in use. In the interviews, I asked all groups if Calico interfered with their existing design habits, if they found the features intrusive, and if using Calico led to any positive changes in the way they design. In the cognitive dimensions analysis, I perform a detailed review of Calico Version Two with respect to each dimension. I consider Calico's potential to support each dimension in comparison to the whiteboard, and review how each dimension manifested within the observed design sessions.

%\subsection{Interviews}
%
%% Refere to marian's book chapter
%In this section, I review the cohesion of the features together, i.e., how well each feature supports one another. I review the findings in Chapter \ref{chapter:evaluation}, reporting on what combinations of features that groups found helpful. For each feature, I review the support they received from other features. 
%
%The results are summarized in Table \ref{table:cohesiveness}. The table can be interpreted as follows, the row header represents the feature under review, the column header represents the feature which supports it, and the intersecting cell represents the groups which found that the feature in the column header did indeed support the feature in the row header. Within the table, ``OSS'' represents the OSS group, ``IxD'' represents the interaction design group, and ``Res'' represents the research group. As an example, for the row header ``Basic sketching'' and the column header ``Scraps'', both the OSS group and the research group found that basic sketching was improved by the scraps feature. 
%
%Each feature was supported as follows:
%
%\begin{table}
%\centering
%\caption{Cohesiveness of features: Degree to which features supported one other.}
%%\begin{tabular}{|c|c|c|c|c|c|}
%%\begin{tabular}{ |p{2cm}|p{2cm}|p{2cm}|p{2cm}|p{2cm}|p{2cm}|}
%\begin{tabular}{ |p{2cm}|p{2cm}|p{2cm}|p{2cm}|p{2cm}|p{2cm}|}
%\hline
%&\multicolumn{5}{c|}{\textit{supported by feature within group}} \\
%\hline
%&Basic sketching &Scraps &Palette &Intentional interfaces & Fading highlighter	\\[5ex]
%\hline
%Basic sketching &  &	OSS, Res &OSS, IxD, Res &OSS, Res & OSS, Res	\\[5ex]
%\hline
%Scraps & OSS, Res &  & OSS, IxD, Res &  & OSS, Res	\\[5ex]
%\hline
%Palette & OSS, IxD, Res & OSS, IxD, Res &  &  &	\\[5ex]
%\hline
%Intentional interfaces&  &  & OSS, IxD, Res &	 & OSS, Res	\\[5ex]
%\hline
%Fading highlighter& 	 &	 &	 &	 & \\[5ex]
%\hline
%\end{tabular}
%\label{table:cohesiveness}
%\end{table}


%\section{Other findings}
%\label{chapter:discussion:other-findings}
%
%//Section removed for the time being.
%\subsection{Context of Calico within the software design process}
%
%Calico played a complementary role to existing tools and practices for each team, serving as an external thinking space for individual and collaborative design. In each case of each team, the members turned to Calico to work on ``wicked problems'' in which the solution was not readily apparent, and required some exploration of the design space before arriving at the solution. Sometimes these problems involve trying to make sense of raw data before a software system is built, as the interaction designers did. Sometimes it is to make modifications to an existing system, as OSS group developers did. Other times they may be building a new system from scratch, as with the research group. Given that the studies each provided glimpses into different pieces of the lifecycle for each group's project, it is difficult to draw a direct comparison of the differences between the approaches of each group. However, several commonolities arose in Calico's role throughout its use for each team. Calico use was not restricted to any single phase of design, but was flexible enough to support groups at different points in their design process.
%
%Calico proved to be a helpful tool for mentally processing ``uncooked'' ideas and information. Individuals used Calico to investigate concepts, discover relationship between data, and also to provide structure over the data. Members from both the OSS group and the research group began some of their design sessions by listing entities, and refining those entities into box-and-arrow diagrams. They arrived at mockups of user interfaces and software structures by performing some or all of the indentified design behaviors, such as refining their sketches, navigating between multiple perspectives of sketches, and explaining their designs to one another. The interaction designers began with pictures of faces of the fifty people they interviewed, and developed categorizations over interacting with those images to produce personas. Both the OSS group and researchers also used Calico to iterate on existing solutions by importing screenshots of source code and screenshots of their user interface.
%
%While Calico was considered a provisional environment, it served as an informal stepping stone to enacting change. Design decisions and plans of action were proposed while using Calico in meetings, but were not solidified until those decisions were recored into a formal document, or being implemented by a developer. By remaining provisional, developers and designers could use Calico as a ``playground'' to explore ideas without commitment, and do so more quickly without needing to worry about precision or completeness. The impromptu, lightweight, and flexible nature of freeform sketching, and by extension Calico, make it a medium which is easier to explore concepts. 
%
%From these assertions, there are several benefits that Calico could provide within the software design process. First, in order to support the processing of ``uncooked'' ideas, Calico could improve its ability to import data. Importing images has already shown to be useful. This ability could be pushed further by pointing Calico directly at other sources of information, such as source code repositories. Second, given that design decisions are proposed within Calico and recorded elsewhere, there is an opportunity to provide a tracibility back to Calico. While the reserach presented showed that sketches will not remain consistent with the most recent decisions because they need to be manually updated, it in fact may be a desirable quality because it reveals past history. Providing traceability to old designs in Calico may reveal design history that would otherwise not be available. Lastly, retaining the history of sketches may be helpful to help designers avoid repeating the same mistakes.
%
%--> Something about physical objects here.

%//Section removed for the time being.
%\subsection{The role of multiple devices}
%
%An important enabling aspect of Calico was that it allowed everyone in a group to sketch on the whiteboard at the same time. The electronic whiteboards had the limitation of only allowing one person to write at a time, which was reported as inhibiting in Chapter \ref{chapter:calico-version-one}, but other users could contribute from their own laptop. This was addition, as past research has shown that groups of individuals will contribute significantly more ideas to a discussion than if only one person can write at a time [Shih 2010].
%
%Individuals across all groups participating in a shared session with their own device reported a greater ability to participate in sessions in comparison to their previous meetings. In some cases, such as the interaction design group, team members continued to not edit content themselves. The interaction designer group previously reported that in a typical case, one member draws on the whiteboard and other provides verbal feedback. While with Calico this interaction remained largely the same, in select cases the member that provided verbal feedback participated from their laptop by copy and pasting his notes into Calico using text-scraps. The OSS group and the research group, however, reported larger benefits. The OSS group reported that having several tablets liberated people from needing to focus on the same content as the speaker, and could diverge into their own sketches, reference past sketches, and participate in sketching without reaching over the should of the person at the large electronic whiteboard.
%
%Calico enabled individuals share the same space more easily. In the case of the OSS group, team members were able to share the same space to carry on their design sessions concurrently using different devices. Members reported that they moved between using Calico on the large electronic whiteboard, their computer, and the tablets. For example, two individuals the large electronic whiteboard, while another developer concurrently used a tablet at his desk. The research group also shared the same meeting space to conduct multiple projects using Calico.
%
%Individuals also prepared content in Calico ahead of time on their own computers. Members from the OSS group and research group copy and pasted content into Calico, and later moved to the large electronic whiteboard to sketch over their imported artifacts. Artifacts ranged from source code, screenshots of user interfaces, and powerpoint slides. 

%%//This section is removed for the time being
%\subsection{Preserving context}
%
%A general fallacy of informal tools, such as whiteboards, is that the context of the artifacts produced is not preserved. The transient nature of the whiteboard results in previous iterations becoming lost as they are manipulated in place or erased. Much of the rationale behind the final iteration is often not recorded in the sketch itself, but instead resides only in the heads of the designers present in the meeting. The sketches themselves become cryptic to anyone who did not create the sketch, and eventually become cryptic to even the original sketcher if enough time elapses to forget.
%
%Members from the OSS group, interaction design group, and research group all remarked that Calico, and to some extent, intentional interfaces, helped them recover past rationale by illuminating details from their past sessions. They deleted fewer canvases, which provided them with more sketches to reference. Rather than creating new iterations in-place and overwriting past iterations, they iterated on their design on newer canvases, establishing a history that would not have been saved on whiteboards. With more canvases, members from all groups had more sketches available to remind them of design paths explored. In the case of the OSS group and the research group, they created chains of canvases that provided further context. The chains served to group canvases into design sessions, and also provide an ordering to the canvases, which either took on the meaning of reminding the sketchers of the chronology of the canvases within the design exploration, or allowed the canvases to build on the content of the previous canvases. Members from the OSS group and research group both used these ordering, and tags, to provide cues to the meaning of content within the canvases. Most canvases were tagged as a ``continuation'', but they found the ``alternative'' tag helpful in explicitly marking a canvas as such. They also remarked that naming canvases was important in remembering their purpose.
%
%Despite the affordances that Calico and the intentional interfaces did provide, users remarked that there were several moments in which they wanted to declare more details, but were unable to do so. When linking canvases by tagging, members of the OSS group and research group wanted to provide direct links between canvases. For example, they wanted to label three canvases as being an alternative of the same canvas. Or, they desired to use multiple tags on a canvas. Further, design was not always linear, but rather a canvas may have references that lead to several other canvases. For example, the OSS group depicted a set of components, and defined the detail of those components in later canvases. Other group members desired to insert notes and annotations within canvases, but did not do so because there was not enough space.
%
%Some of the context of design sessions could not be captured due to the limitations of the technology. Reviews confirmed that much of the design that takes place with Calico happens by talking over designs, pointing at figures, and gesturing to nothing in particular. One remote collaborator reported that being able to see participants using video conferencing tools was important because it allowed him to see the gestures of others. In the OSS group, usage logs showed that users viewed canvases without interacting with it for long periods of time, and interviews confirmed that developers were using the image to discuss the next steps. The sketch played an important role, but no evidence of this was evident from Calico without asking the developers. In a similar situation, usage logs demonstrated a intensive periods of usage for the highlighter in the OSS group, but upon playing back a slideshow of the usage, members of the OSS group could not recall the topic they used it to discuss.
%
%In addition to the information available in Calico, the usage logs contained information that may have provider deeper context about sketches in future versions of Calico. The usage log provided intermediate screenshots of sketches, which revealed unique representations that provided information about how they led up to a final design. The logs further provided summaries of sessions, reporting when sketches were created, and the amount of time spent in creating the sketch. 

\section{Summary}
\label{discussion:summary}

Having presented my analysis of the experiences reported in Chapter \ref{chapter:evaluation}, I now return back to the research question in Chapter \ref{chapter:research-question}. The goal of Calico was to create a minimally invasive, small set of features that worked together to address the full set of design behaviors. Rather than create a new feature for each design behavior, I created a small set of four features that address all fourteen. What I have is a small set of features which work together to address all design behaviors.

Addressing the first part of the research question, the observations made thus far show that Calico's features were indeed minimally invasive in real world  design activities. Minor issues did exist that users found invasive, for example limited space in a canvas, poor quality of handwriting on the large electronic whiteboards, and slow movement speed of image scraps for the interaction designers. Despite these issues, all users were able to use Calico to perform actual work in real design sessions, and users saw Calico as improving their capability to design in comparison to the regular dry-erase whiteboard.

Addressing the second part of the research question, most of the features did coherently work together. Table \ref{table:discussion:minimally-invasive} succinctly summarizes the contributions of each feature to the set of fourteen design behaviors. Each feature supports multiple design behaviors, and they clearly are integrated, and few of the features stand alone. Scraps and the palette are very much integrated, where both work together in design behaviors 2, 4, 6, and 14. Scraps were also integrated with intentional interfaces, both supporting design behaviors 5, 6, 7, and 9. The fading highlighter, however, stands apart from the other features, supporting design behaviors 8 and 13. While it stands alone, it is powerful in what it does, and does so in a minimal and lightweight fashion.

\begin{center}
\begin{longtable}{|p{4cm}|p{4cm}|p{8cm}|}
\caption{The set of design behaviors and the features that support them}\\
\hline
\textbf{Design Behavior} & \textbf{Supporting Feature} & \textbf{Design Principles} \\
\hline
\endfirsthead
\multicolumn{3}{c}%
{\tablename\ \thetable\ -- \textit{Continued from previous page}} \\
\hline
\textbf{Behavior} & \textbf{Supporting Feature} & \textbf{Design Principles}\\
\hline
\endhead
\hline \multicolumn{3}{r}{\textit{Continued on next page}} \\
\endfoot
\hline
\endlastfoot
1. They draw different kinds of diagrams	&Scraps \& connectors	& + Supported lists, box-and-arrow, UI diagrams well\\
\hline
\multirow{2}{4cm}{2. They produce sketches that draw what they need, and no more}&Scraps \& connectors	& + Represented software components as scrap and connectors, very little detail included. \\
\cline{2-3}
&Palette 	& + Reused low detail scraps they created across canvases\\
\cline{1-3}
3. They refine and evolve their sketches over time	&Scraps and connectors& + Plane sketches could be made into scraps, have detail added to them, and add connectors\\
\cline{1-3}
4. They use impromptu notations	&Scraps \& connectors	& + Color took on meaning in connectors

+ Scraps were tagged/underlined with color\\
\cline{2-3}
	&Palette	& + Visual icons were created using scraps and used repeatedly\\
\hline
%	&Scraps and connectors &Elementary visual looks and behavior can be randomly composed	&\\
%\hline
\multirow{2}{4cm}{5. They move from one perspective to another}&Scraps \& connectors	& + Different notational conventions can be mixed and matched on a single canvas
\\
\cline{2-3}
	&Intentional Interfaces	& + Users can explicitly request a new canvas to work on a perspective

+ Canvases are linked and ordered, helping find related perspectives	
	\\
\hline
6.      They move from one alternative to another	&Scraps \& connectors	& + Different alternatives can be quickly constructed by copying and moving and otherwise manipulating scraps and connectors\\
\cline{2-3}
	&Palette	& + Different alternatives can be quickly constructed by reusing elements from the palette and composing them differently \\
\cline{2-3}
	&Intentional Interfaces	& + Users can explicitly request a new canvas to work on a different alternative

+ Canvases are linked and ordered, helping find related alternatives	\\
\hline
7.      They move from one level of abstraction to another	&Scraps \& connectors	& + Different abstractions can be quickly constructed by copying and moving and otherwise manipulating scraps and connectors	\\
\cline{2-3}
	&Intentional Interfaces	& + Users can explicitly request a new canvas to work on a deeper level of abstraction

+ Canvases are linked and ordered, helping find related canvases of different levels of abstraction	\\
\cline{1-3}
8.      They perform mental simulations	& Fading highlighter	& + Users can use the highlighter to mark up their diagrams without editing them	\\
\hline
9.      They juxtapose sketches	&Scraps	& + Uses can move perspectives, alternatives, and abstractions next to one another by moving scraps\\
\cline{1-3}
10.  They review their progress	&Intentional Interfaces	& + Users can step back and examine their progress and process, overall and in parts, in the intention view\\
\hline
11.  They retreat to previous ideas	&Intentional Interfaces	& + Users can choose to enter one canvas in the intentional view or make a new canvas at any time\\
\hline
12.  They switch between synchronous and asynchronous work	&Intentional Interfaces	& + Users can choose to enter one canvas in the intention view, or make a new canvas and work separately\\
\hline
13.  They explain their sketches to each other	& Fading highlighter	& + Users can use the highlighter to draw attention to certain parts of a canvas	\\
\hline
14.  They bring their work together	&Scraps \& connectors	& + Scraps can pick up sketches and drop those sketches onto other scraps, merging them\\
\cline{2-3}
	&Palette	& + Designers can place sketches from different canvases into palette and later merge them into a single canvas
\label{table:discussion:minimally-invasive}
\end{longtable}
\end{center}

%Do these features fit together, and do I believe that they fit together
Addressing the third part of the research question, the set of four features did indeed support all design behaviors. Scraps and the palette supported designers in the kinds of sketches they created (design behaviors 1 -- 4). They allowed diagrams to ``grow organically'', have rich coversations over the runtime behavior of components, and build relationships over unrelated sets of data. Intentional interfaces, scraps, the palette, and the fading highlighter helped support the way designers navigated between sketches (design behaviors 5 -- 11). Intentional interfaces helped designers find relationed perspectives, alternatives, and levels of abstraction, and did so by attaching related canvases with tagged links and providing an order to them, which also helped to review their progress and retreat to previous ideas. The fading highlighter further supported designers in mentally simulating over box-and-arrow diagrams. Lastly, Intentional interfaces, the fading hihglighter, scraps and the palette helped designers in collaborating with one another (design behaviors 12 -- 14). Intentional interfaces helped designers in moving between synchronous work and opportunistically branching off to asynchronous work. The fading highlighter enabled designers to explain their sketches to one another from remote computers. The palette further allowed designers to bring work together from different canvases, and scraps also help to coordinate bringing work together.


%Analysis of grid versus intentional interfaces. We reduced the number of features that used to be there where each serve multiple purposes.

%\section{Limitations and threats to validity}

%--> What do these sessions allow me to say? The sessions are not directly comparable.

%
%In this section, I address potential issues which may lead to threats in validity.
%
%- Was not able to observe studies in person.
%- People interviewed may not have remembered correctly, or incorrectly interpreted logs.
%- All uses of Calico may have been idiosyncratic, and specific to the culture observed.
%- Observed different phases of the lifecycle for each group.

%%% Local Variables: ***
%%% mode: latex ***
%%% TeX-master: "thesis.tex" ***
%%% End: ***
